
\documentclass[12pt]{article}
\linespread{1.1}
\pagestyle{myheadings}\markboth{Pseudo-code: \today}{Pseudo-code: \today}

\usepackage{authblk}
\usepackage{amsfonts}
\usepackage{amssymb}
\usepackage{amsmath}
\usepackage{amsthm}
\usepackage{framed}
\usepackage{url}
\usepackage{enumitem}
\usepackage{geometry}
\geometry{left=2cm, right=2cm, top=2cm, bottom=2cm}


\usepackage{booktabs}
\newcommand{\ra}[1]{\renewcommand{\arraystretch}{#1}}
\usepackage{multirow}

\usepackage{mdframed}
\usepackage{mathtools}
\usepackage{algorithm}  
\usepackage{algorithmic} 

\renewcommand{\algorithmicrequire}{\textbf{Input:}} 
\renewcommand{\algorithmicensure}{\textbf{Output:}}
\renewcommand{\algorithmicforall}{\textbf{for each}}

\newtheorem{theorem}{Theorem}
\newtheorem{lemma}[theorem]{Lemma}
\newtheorem{corollary}[theorem]{Corollary}
\newtheorem{conjecture}[theorem]{Conjecture}
\newtheorem{observation}[theorem]{Observation}
 
\newcommand{\dmax}{d_\mathrm{max}}

\newcommand{\intt}{\mathrm{int}}
\newcommand{\inn}{\mathrm{in}}
\newcommand{\ex}{\mathrm{ex}}
\newcommand{\co}{\mathrm{co}}

\newcommand{\ext}{\mathrm{ex}}

\newcommand{\typ}{\mathrm{t}}

\newcommand{\ce}{\mathrm{ce}}
\newcommand{\ac}{\mathrm{ac}}
\newcommand{\bt}{\mathrm{bt}}
\newcommand{\ft}{\mathrm{ft}}
\newcommand{\bd}{\mathrm{bd}}
\newcommand{\bh}{\mathrm{bh}}
\newcommand{\bn}{\mathrm{bn}}
\newcommand{\bl}{\mathrm{bl}}% {\mathrm{b\ell}}
\newcommand{\bc}{\mathrm{bc}}
\newcommand{\Bc}{\mathrm{Bc}}
 
\newcommand{\tail}{{\mathrm{tail}}}
\newcommand{\head}{{\mathrm{head}}}
 
\newcommand{\ntreeS}{n_{\mathrm{tree}}^{\mathrm{S}}}
\newcommand{\ntreeT}{n_{\mathrm{tree}}^{\mathrm{T}}}

\providecommand{\MSG}{\mathrm{MSG}}
\providecommand{\true}{\mathtt{True}}
\providecommand{\false}{\mathtt{False}}

\renewcommand{\d}{${\rm d}$}
\newcommand{\prt}{${\rm prt}$}
\newcommand{\V}{{\rm V}}
\newcommand{\E}{{\rm E}}
\newcommand{\G}{\mathcal{G}} 


\newcommand{\ta}{{\tt a}}
\newcommand{\tb}{{\tt b}}
\newcommand{\tB}{{\tt B}}
\newcommand{\tC}{{\tt C}}
\newcommand{\tO}{{\tt O}}
\newcommand{\tN}{{\tt N}}
\newcommand{\tS}{{\tt S}}
\newcommand{\tF}{{\tt F}}
\newcommand{\tSi}{{\tt Si}}
\newcommand{\tCl}{{\tt Cl}}
\newcommand{\tI}{{\tt I}}
\newcommand{\tHg}{{\tt Hg}}

\newcommand{\f}{\pmb{f}}
\newcommand{\w}{\pmb{w}}
\newcommand{\vv}{\pmb{v}}
%\newcommand{\u}{\pmb{u}}
%\newcommand{\y}{\pmb{y}}
\newcommand{\y}{\pmb{y}}
\newcommand{\x}{\pmb{x}}
\newcommand{\z}{\pmb{z}}
\newcommand{\s}{\pmb{s}}
\newcommand{\1}{\pmb{1}}
\newcommand{\0}{\pmb{0}}
\newcommand{\rms}{\pmb{rm}}


\newcommand{\val}{\mathrm{val}}
\newcommand{\dia}{\mathrm{dia}}
\newcommand{\nec}{\mathrm{nec}}
\newcommand{\Tl}{\mathrm{Tl}} 
\newcommand{\LG}{\mathrm{\Lambda\cup \Gamma}} 
\newcommand{\pred}{\mathrm{pred}}
\newcommand{\mul}{\mathrm{mul}}
\newcommand{\smp}{\mathrm{smp}}
\newcommand{\inl}{\mathrm{inl}}
\newcommand{\en}{\mathrm{end}}
\newcommand{\pair}{\mathrm{pair}}

\newcommand{\LGB}{\mathrm{\Lambda\cup \Gamma\cup \Bc}}
\newcommand{\LtoD}{\mathrm{\Lambda\cup \Gamma\cup \Bc\cup \Dg}}
 
\newcommand{\FT}{\mathcal{FT}} 
\newcommand{\T}{\mathcal{T}}  
\newcommand{\W}{\mathrm{W}} 
\newcommand{\Ww}{\mathcal{W}}
\newcommand{\Vv}{\mathcal{V}}
\newcommand{\I}{\mathrm{I}} 
\newcommand{\Pa}{\mathrm{P}} 
\newcommand{\SEQ}{\mathrm{SEQ}} 
\newcommand{\Ds}{\mathrm{Ds}} 
\newcommand{\fq}{\mathrm{fq}} 
\newcommand{\Fq}{\mathrm{Fq}} 
\newcommand{\col}{\mathrm{col}} 
\newcommand{\nb}{\mathrm{nb}}
\newcommand{\Dg}{\mathrm{Dg}}
\newcommand{\dg}{\mathrm{dg}}

\renewcommand{\d}{\mathrm d}
%\newcommand{\col}{\mathrm col}
%\newcommand{\prt}{\mathrm {prt}}

%\newcommand{\pred}{{\rm pred}}

\begin{document}  
 
\begin{center}
 {\Large\bf Pseudo-codes for Graph Search Algorithm}
\end{center} 
For each base vertex $v$ or edge $e$, we are given a set
$\mathcal{F}$ of fringe trees. 
For each $T \in \mathcal{F}$, \\
- identify the root information
such as root label, multiplicity and degree. 
Let $\ta, m, d$ be label, multiplicity and degree of the root of 
the tree $T$. \\
- obtain $T[+\Delta]$ and vector $\w$ of 
$T[+\Delta]$ 
for each $\Delta \in [1, \val(\ta) - d]$ and 
include $\w$ in the respective sets
$\V_{\en}^{(0)}(\ta, d, m; \x^*), V_{\inl}^{(0)}(a, d, m; \x^*), 
\V_{\co+\Delta}^{(0)}(a, d, m, h; \x^*)$, $h \leq 2$, 
$\V_{\co+(\Delta+1)}^{(0)}(a, d, m, p; \x^*)$


 

% \newpage
% % % % % % % % % % % % % % % % % % % % % % % % % % % % % % % % % % % % % % % % % 
% 
%                 Getting Half-portions
% 
% % % % % % % % % % % % % % % % % % % % % % % % % % % % % % % % % % % % % % % % % 

\section{Computing Frequency Vectors of End-Subtrees }
%
%We here give an outline of procedures 
%to generate the sets of frequency vector of bi-rooted nc-trees. 
For an integer $h \geq 1$, element 
$\ta \in \Lambda$,
integers $d  \in [1,\val(\ta)-1]$, and 
$m\in [d,\val(\ta)-1]$ 
 we give a procedure to 
compute the set $\V_{\en}^{(h)}(\ta, d, m; \x^*)$.
\bigskip\\
\noindent {\sc ComputeEndSubtreeOne($\ta, d, m, h$)}
\begin{tabbing}
%
% First, we set a tab stop every 3 mm
\hspace{3mm} \= \hspace{3mm} \= \hspace{3mm} \= \hspace{3mm} \= %
\hspace{3mm} \= \hspace{3mm} \= \hspace{3mm} \= \hspace{3mm} \= %
\hspace{3mm} \= \hspace{3mm} \= \hspace{3mm} \= \hspace{3mm}  \kill
% Throughout the description we use \+ and \- to advance and reduce by one tab stop
% the indent of the following lines, and by necessity, use \> to advance the indent 
% of the current line
%
{\bf Input:}
Element $\ta \in \Lambda$, 
integer $d \in[1,\val(a)-1]$, 
$m \in [d, \val(a)-1]$, $h \geq 1$. \\
\hspace{3mm}  /* Global data: A vector $\x^* = (\x^*_{\intt}, \x^*_{\ex}, b)$ 
%with 
%$\x^*_{\co} \in \mathbb{Z}^{\Lambda^{\co}}$, 
%$\x^*_{{\tt t}} \in \mathbb{Z}^{\Lambda^{{\tt t}}}$,
%${\tt t} \in \{\inn, \ex\}$, 
\+ \\
a non-negative integer $b$,
                 the collection  \\
                 $\Vv_{\inl}$ vector sets 
                 $\V_{\inl}(\ta, d-1, m_{\ta}; \x^*)$, 
                 %$d_{\ta} \in [0, \val(\ta)-3]$,
                 $m_{\ta} \in [d-1, \val(\ta) -2]$\\
                 $\Vv_{\en}^{(h-1)}$ of vector sets  
                 $\V_{\en}^{(h-1)}(\ta_1, d_1, m_1; \x^*)$, 
                 $\ta_1 \in \Lambda$,
                  $d_1 \in [1, \val(\ta_1)-1]$, \\
                  $m_1 \in [d_1, \val(\ta_1) - 1]$. */ \- \\
{\bf Output:} The set $\V_{\en}^{(h)}(\ta, d, m; \x^*)$, where 
we store these vectors in a trie. \\
$\W := \emptyset$; \\
{\bf for each} triplet  $(\tb, d_{\tb}, m_{\tb})$ {\bf do} \+ \\
%with $\tb \in \Lambda$, $d_{\tb} \in [1, 3]$ {\bf do} \+ \\
    {\bf for each} triplet $(\ta, d-1, m_{\ta})$ {\bf do} \+ \\
    %{\bf if} $\delta = 1$ \+ \\
    %{\bf for each} $m' \in [1, 3]$ such that \+ \+ \\ 
    {\bf for each} $\y^{\tb} = (\y^{\tb}_{\intt}, 
    \y^{\tb}_{\ex}, 0) \in 
    \V_{\en}^{(h-1)}(\tb, d_{\tb}, m_{\tb}; \x^*)$
                         {\bf do} \+ \\
            {\bf for each} $m' \in [1, 3]$ such that \+ \+ \\ 
                      - $\gamma^{\intt} = (\ta \{d+1\}, \tb  \{d_{\tb} + 1\},
                       m') \in \Gamma^{\intt}$ and\\
                      - $m_{\ta} + m' = m, 
                      m_{\ta} + m' + 1\leq \val(\ta)$ and 
                      $m' +m_{\tb} \leq \val(\tb)$
                      
                       {\bf do} \- \\
                      {\bf for each} $\y^{\ta} = (\y^{\ta}_{\intt}, \y^{\ta}_{\ex}, 0) \in 
                      \V_{\inl}^{(0)}(\ta, d-1,m_{\ta}; \x^*)$ 
                      {\bf do} \+\\                      
                     $\y_{\intt} := \y^{\ta}_{\intt} + \y^{\tb}_{\intt} + \1_{\gamma^{\intt}}$;  \\
                     $\y_{\ex} := \y^{\ta}_{\ex} + \y^{\tb}_{\ex}$;  
                     $\y:=(\y_{\intt}, \y_{\ex}, 0)$;\\
                     {\bf if } $\y \leq \x^*$ {\bf then} \+ \\  
                                          $\W := \W \cup \{ \y \}$; \- \\              
                      {\bf end if} \- \\
                 {\bf end for} \- \\
            {\bf end for} \- \\
        {\bf end for} \- \\
    {\bf end for} \- \\
{\bf end for}; \\
Output $\W$ as $\V_{\en}^{(h)}(\ta, d, m; \x^*)$.
\end{tabbing}


%%%%%%%%%%%%%%%%%%%%%%%%%
\section{Generating Frequency Vectors of Rooted 
Core-subtrees }
%
%We here give an outline of procedures 
%to generate the sets of frequency vector of bi-rooted nc-trees. 
For an integer $h \geq 1$, element 
$\ta \in \Lambda$, integers
$\Delta \in [2, 3]$, 
 $d  \in [1, \val(\ta) - \Delta]$, and 
$m\in [d,\val(\ta)-1]$ 
 we give a procedure to 
compute the set $\V_{\co + \Delta}^{(0)}(\ta, d, m, h; \x^*)$.
\bigskip\\
\noindent {\sc ComputCoreSubtreeOne($\ta, d, m, h$)}
\begin{tabbing}
%
% First, we set a tab stop every 3 mm
\hspace{3mm} \= \hspace{3mm} \= \hspace{3mm} \= \hspace{3mm} \= %
\hspace{3mm} \= \hspace{3mm} \= \hspace{3mm} \= \hspace{3mm} \= %
\hspace{3mm} \= \hspace{3mm} \= \hspace{3mm} \= \hspace{3mm}  \kill
% Throughout the description we use \+ and \- to advance and reduce by one tab stop
% the indent of the following lines, and by necessity, use \> to advance the indent 
% of the current line
%

%%% Important %% In the next 2 can be replaced by \rho to 
%%% make the algorithm general
{\bf Input:}
Element $\ta \in \Lambda$, 
integer $d \in[1,\val(a)-\Delta]$, 
$m \in [d, \val(a)-1]$, $h \geq 1$. \\
\hspace{3mm}  /* Global data: A vector $\x^* = (\x^*_{\intt}, \x^*_{\ex}, b)$, \+ \\
a non-negative integer $b$,
                 the collection  \\
                 $\Vv_{\co+ \Delta + 1}^{(0)}$  vector sets 
                 $\V_{\co + \Delta  +1}^{(0)}(\ta, d-1, m_{\ta}, 
                 p; 
                 \x^*)$, 
                 %$d_{\ta} \in [0, \val(\ta)-3]$,
                 $m_{\ta} \in [d-1, \val(\ta) -\Delta -1]$, \\
                 $p \in [0, 2 (= \rho)]$\\
                 $\Vv_{\en}^{(h-2-1)}$ of vector sets  
                 $\V_{\en}^{(h-2-1)}(\ta_1, d_1, m_1; \x^*)$, 
                 $\ta_1 \in \Lambda$,
                  $d_1 \in [1, \val(\ta_1)-1]$, \\
                  $m_1 \in [d_1, \val(\ta_1) - 1]$,
                  integer $g \geq 1$. */ \- \\
{\bf Output:} The set $\V_{\co + \Delta}^{(0)}(\ta, d, m, h; \x^*)$, where 
we store vectors $\V_{\co + \Delta}^{(0)}(\ta, d, m, h; \x^*)$, \\
 ~~~in a trie.  \\
$\W := \emptyset$; \\
{\bf for each} triplet  $(\tb, d_{\tb}, m_{\tb})$ {\bf do} \+ \\
%with $\tb \in \Lambda$, $d_{\tb} \in [1, 3]$ {\bf do} \+ \\
    {\bf for each} triplet $(\ta, d-1, m_{\ta}, p)$ {\bf do} \+ \\
    %{\bf if} $\delta = 1$ \+ \\
    %{\bf for each} $m' \in [1, 3]$ such that \+ \+ \\ 
    {\bf for each} $\y^{\tb} = (\y^{\tb}_{\intt}, \y^{\tb}_{\ex}, 0) \in 
    \V_{\en}^{(h-2-1)}(\tb, d_{\tb}, m_{\tb}; \x^*)$
                         {\bf do} \+ \\
            {\bf for each} $m' \in [1, 3]$ such that  \\ 
                      	~~~- $\gamma^{\intt} = (\ta \{d+\Delta\}, \tb  \{d_{\tb} + 1\},
                       m') \in \Gamma^{\intt}$ and\\
                      ~~~- $m_{\ta} + m' = m, 
                      m_{\ta} + m' + \Delta \leq \val(\ta)$ and 
                      $m' +m_{\tb} \leq \val(\tb)$
                      
                       {\bf do} \+ \\
                      {\bf for each} $\w^{\ta} = (\w^{\ta}_{\intt}, \w^{\ta}_{\ex}, 0) \in 
                      \W_{\inl}^{(0)}(\ta, d-1,m_{\ta}, p; \x^*)$ 
                      {\bf do} \+\\                      
                     $\w_{\intt} := \w^{\ta}_{\intt} + \y^{\tb}_{\intt} + \1_{\gamma^{\intt}}$;  \\
                     $\w_{\ex} := \w^{\ta}_{\ex} + \y^{\tb}_{\ex}$;  
                     $\y:=(\y_{\intt}, \y_{\ex}, 1)$;\\
{\bf if } $\y \leq \x^*$ {\bf then} \+ \\  
                                          $\W := \W \cup \{ \y \}$; \- \\              
                      {\bf end if} \- \\
                     %{\bf end if} \- \\
                 {\bf end for} \- \\
            {\bf end for} \- \\
        {\bf end for} \-\\
    {\bf end for} \-\\
{\bf end for};  \\
Output $\W$ as $\V_{\co + \Delta}^{(0)}(\ta, d, m, h; \x^*)$.
\end{tabbing}

%%%%%%%%%%%%%

% \newpage
% % % % % % % % % % % % % % % % % % % % % % % % % % % % % % % % % % % % % % % % % % 
% 
%     Computing DAG
% 
% % % % % % % % % % % % % % % % % % % % % % % % % % % % % % % % % % % % % % % % % % 
\section{Computing DAG Representation for $v$-Components}

%For an integer $h \geq 1$, element 
%$\ta \in \Lambda$,
%integers $d  \in [1,\val(\ta)-1]$, and 
%$m\in [d,\val(\ta)-1]$ 
% we give a procedure to 
%compute the set $\V_{\en}^{(h)}(\ta, d, m; \x^*)$.
%\bigskip\\
\noindent {\sc DAGRepresentationVertex($\ta_v, d_v, m_v, t,\Delta_v,  \x^*$)}
\begin{tabbing}
%
% First, we set a tab stop every 3 mm
\hspace{3mm} \= \hspace{3mm} \= \hspace{3mm} \= \hspace{3mm} \= %
\hspace{3mm} \= \hspace{3mm} \= \hspace{3mm} \= \hspace{3mm} \= %
\hspace{3mm} \= \hspace{3mm} \= \hspace{3mm} \= \hspace{3mm}  \kill
%
{\bf Input:}
%Element $\ta \in \Lambda$, 
%integer $d \in[1,\val(a)-1]$, 
%$m \in [d, \val(a)-1]$, $h \geq 1$. \\
\hspace{3mm}  /* Global data: A vector $\x^* = (\x^*_{\intt}, \x^*_{\ex}, b)$, \+ \\
a non-negative integer $b$,\\
integers $t$, 
$\Delta_v \in [2, 3]$,  
element
$\ta_v \in \Lambda$,  \\
integers
$d_v \in [0, \val(\ta_v) - \Delta_v-1]$, 
$m_v \in [d_{v}, \val(\ta_v) - \Delta_v-1]$, \\
                 the collection  
                 $\Vv^{(0)}_{\inl}$ vector sets 
                 $\V^{(0)}_{\inl}(\ta, d, m; \x^*)$, \\
                 %$d_{\ta} \in [0, \val(\ta)-3]$,
                 %$m_{\ta} \in [d-1, \val(\ta) -2]$\\
                 an integer $t \geq 0$, \\
                 $\Vv^{(h)}_{\en}$ of vector sets 
                 $\V_{\en}^{(h)}(\ta_1, d_1, m_1; \x^*)$, 
                 $\ta_1 \in \Lambda$,
                  $d_1 \in [1, \val(\ta_1)-1]$, \\
                  $m_1 \in [d_1, \val(\ta_1) - 1]$, 
                  $1 \leq h \leq t$, \\
                  the collection 
                   $\Vv_{\en}^{(0)}$ of sets 
                  $\V_{\en}^{(0)}(\ta_1, d_1, m_1; \x^*)$, 
                  $\ta_1 \in \Lambda$,
                  $d_1 \in [1, \val(\ta_1)-1]$, \\
%                  $m_1 \in [d_1, \val(\ta_1) - 1]$,\\
%                 the collection $\Vv^{(h)}_{\inl+\Delta}$ of vector sets  
%                $\V_{\inl+\Delta}^{(h)}
%                (\tb, d_{\tb}, m_{\tb}; \x^*)$, \\
%                 $\tb \in \Lambda$,
%                  $d_{\tb} \in [1, \val(\tb)-1]$, 
%                  $m_{\tb} \in [d_{\tb}, \val(\tb) - 1]$, 
%                  $1 \leq h \leq \delta_1$, \\
                  the collection $\Vv_{\co+(\Delta_v + 1)}^{(0)}$ of sets 
                 $\V_{\co+(\Delta_v + 1)}^{(0)}(\ta_v, d_{v} -1, 
                 m'', p; \x^*)$
                  for $p \leq 2$.                 
                  */ \- \\
{\bf Output:} A vertex-labeled and edge-labeled DAG representation.\\
%
$F:= \emptyset$;\\
$G := (N, A)$; $A:= \emptyset$; 
$N:= \emptyset;$\\
%{\bf if } $t \geq 2$ {\bf then} \+ \\%1
	{\bf for each} $\w \in \V_{\co+(\Delta_v + 1)}^{(0)}(\ta_v, d_{v} -1, 
                 m', p; \x^*)$ for each possible 
                $(m', p)$ {\bf do} \+ \\%2
                {\bf for each} $\y \in 
                \V_{\en}^{(t)}(\ta_1, d_1, m_1; \x^*)$ 
                for each possible $(\ta_1, d_1, m_1)$ {\bf do}\+ \\%3
                % /* Test if $\y_i, i=1, 2$ is a feasible pair */ \\
                 {\bf if} there exists  
                 $\gamma:=(\ta_{v}\{d_v+\Delta_v\}, \ta_1\{d_{1}+1\}, m_v - m')
                  \in \Gamma^{\co}$\\
%                   with \\
% ~~~~~$m'' \in [1, \min\{3, \val(\ta)-m_{1}, \val(\tb)-m_{\tb}\}]$
 ~~~~~such that  $ \y + \w + \1_{\gamma} = \x^*$ {\bf then}
 \+ \\ % 4
 		%$N: = N \cup \{ (\y, t-3; \tb, d_{\tb}, m_{\tb} )$;
 		%\\
 		$N : = N \cup \{ (\x^*, t+1; 
 		\ta_v, d_v, m_v)\}$;\\
 		$N : = N \cup \{ (\y, t; 
 		\ta_1, d_1, m_1)\}$; \\
 		$A:= A \cup \{a_{\x^*\y}\}$ and\\
 	 let the label of the arc 
 		$a_{\x^*\y}$ to be $(\w, m_v - m')$; \- \\
% 		$F: = F \cup \{(\x^*, \y; \w, m_v-m'; \ta_v, d_v-1, m_v; 
% 		\ta_1, d_1, m_1)\}$\- \\
				{\bf end if}\- \\ %4  
				{\bf end for} \- \\ %3 end of y1
	{\bf end for}; \\ %2 end of y1
%	$G' := G$;\\
 %%%%%
	{\bf for each} $\ell \in (t, \ldots, 1)$ {\bf do}\+ \\ % 4
		$G'':=(N'', A'') :=$ DAGSublayer($\Vv_{\en}^{(\ell-1)}, G,
		 \ell-1, \Vv^{(0)}_{\inl}$);\\
		~~~$N:= N \cup N''; A:= A_2 \cup A''$\- \\		
	{\bf end for}; \\
	%
%	$G' := G_1$;\\
%	{\bf for each} $\ell \in (\delta_1, \ldots, 1)$ {\bf do}\+ \\ % 4
%		$G'':=(N'', A'') :=$ DAGSublayer($\Vv_{\inn +\Delta}^{(\ell-1)}, G', \ell-1, \Vv^{(0)}_{\inl}$), \\
%		~~~where we use 
%		$\V_{\co+(\Delta + 1)}^{(0)}(\ta_v, d_{v} -1, 
%                 m'', p; \x^*)$ when $\ell = \delta_1$;\\
%		~~~$N_1:= N_1 \cup N''; A_1:= A_1 \cup A''$\- \\		
%	{\bf end for} \\
	
%	{\bf else} /* $\delta_1 + \delta_2 + 1 = 1$*/\+  \\
%	%
%	 This case can be handled analogously to the above case,
%	 \\
%	 ~~~where we take 
%	 $\y_1 \in \V_{\co+(\Delta + 1)}^{(0)}(\ta_v, d_{v} -1, 
%                 m'', p; \x^*)$ and 
%                 $\y_2 \in \V_{\en}^{(0)}(\ta_1, d_1, m_1; \x^*)$ \- \\
%{\bf end if}; \\ %1 end of if
Output $G$ as DAG representations and the set 
$F$ of feasible pairs of $v$-component.
\end{tabbing}
%%%%%
%%%%%
%%%%%
\bigskip
\noindent {\sc DAGSublayer($\Vv, G, \ell, \Vv'$)}
\begin{tabbing}
%
% First, we set a tab stop every 3 mm
\hspace{3mm} \= \hspace{3mm} \= \hspace{3mm} \= \hspace{3mm} \= %
\hspace{3mm} \= \hspace{3mm} \= \hspace{3mm} \= \hspace{3mm} \= %
\hspace{3mm} \= \hspace{3mm} \= \hspace{3mm} \= \hspace{3mm}  \kill
{\bf Input:} 
A family $\Vv$ of set of vectors of trees 
with root label $\ta_1$, degree \+ \\
$d_1$ and multiplicity $m_1$, 
$G = (N, A)$,  \\
a family of   
                 $\Vv'$ vector sets of fringe-trees, \\
                 %$\V_{\inl}(\ta, d, m; \x^*)$, \\
                 $\ell$ (the height of the layer  
                 that we add in $G$ at this stage).\- \\
%                 
{\bf Output:}  A DAG $G'$ that is a super-graph of $G$.\\
%
$G':= G$;\\
{\bf for each} $\y_1 \in \Vv$ {\bf do} \+ \\%1
	{\bf for each} $\w \in \Vv'$ {\bf do} \+ \\%2
		{\bf if } there exists $\gamma \in \Gamma^{\inn}$ and 
		some $\y_2 \in N$ such that \\
		~~~$\y_i, i=1,2$ are feasible, i.e., 
		$\y_1 +\w+1_{\gamma} = \y_2$ {\bf then}\+\\%4
%		{\bf for each} $\gamma=(\ta\{d+2\}, \ta_1\{d_{1}+1\}, m'')
%                  \in \Gamma^{\inn}$ with \\
%                 ~~~~~$m'' \in [1, \min\{3, \val(\ta)-m, \val(\tb)-m_{\tb}\}]$ {\bf do}\+ \\ %3
%		 {\bf if} there exists  
				{\bf if} $\y_1 \not\in N$ {\bf then} 
				$N:= N \cup \{(\y_1, \ell;  \ta_1, 
				d_1, m_1)\}$;\\
				$A:= A \cup \{a_{\y_2\y_1}\}$ and \\
			~~~label the arc from 
				$\y_2$ to $\y_1$ by $(\w, m)$, \\
				~~~where $m$ is 
				the bond multiplicity in $\gamma$\- \\
                 
               {\bf end if} \-\\%4    
		%	{\bf end for} \- \\%3
	{\bf end for} \- \\%2
{\bf end for};  \\%1
Output $G'$ as a required DAG.
\end{tabbing}

% \newpage
% % % % % % % % % % % % % % % % % % % % % % % % % % % % % % % % % % % % % % % % % % 
% 
%     Enumerating paths from DAG
% 
% % % % % % % % % % % % % % % % % % % % % % % % % % % % % % % % % % % % % % % % % % 
\section{Enumerating Paths in DAG}
\noindent {\sc EnumPaths(DAG)}
\begin{tabbing}
\hspace{3mm} \= \hspace{3mm} \= \hspace{3mm} \= \hspace{3mm} \= %
\hspace{3mm} \= \hspace{3mm} \= \hspace{3mm} \= \hspace{3mm} \= %
\hspace{3mm} \= \hspace{3mm} \= \hspace{3mm} \= \hspace{3mm}  \kill
%
{\bf Input:} A rooted vertex-labeled and edge-labeled DAG 
$G = (N, A)$.\\
%
{\bf Output:} All directed paths from sources to leaves.\\
We consider $G$ a rooted DAG with a virtual root $r$ that is 
adjacent with all source vertices; \\
We consider dsf ordering on the vertices of $G$ starting from root with index $0$ and \\
~~~traverse $G$ in left-right ordering on the children of each vertex;\\
$\mathcal{P}:= \emptyset$;\\
Let $Q_i :=$ set dfs label of all children of the vertex with dfs label 
$i$, $i \in |N|$;\\
{\bf if} $|Q_1| = 0$ {\bf then} \+ \\
$\mathcal{P}:= \mathcal{P} \cup 
\{1\}$\- \\
{\bf else}\+ \\
	{\bf while} $Q_1 \neq \emptyset$ {\bf do} \+ \\
		Let $i$ be the smallest integer in $Q_1$;\\
		%$P: = \emptyset$; 
		%$P' := P$;\\
		Let $\y_1,$  and $\y_i$ be the label of vertices with 
		dfs label $1$ and $i$, respectively, and\\
		 ~~~the label of arc between $y_1$ and $\y_i$ is 
		 $(\w, m)$\\
		$P:= ((\y_1, \y_i, \w, m))$;\\
		$\mathcal{P'}:=$ PathRecursion$(P, i, \mathcal{P}, G)$;
		$Q_1 := Q_1 \setminus \{i\}$;
		$\mathcal{P}: = \mathcal{P} \cup \mathcal{P'}$ \- \\		
	{\bf end while}\-\\
{\bf end if}; \\
Output $\mathcal{P}$ as the required family of paths.
\end{tabbing}
\bigskip
\noindent {\sc PathRecursion$(P, i, \mathcal{P}, G)$}
\begin{tabbing}
\hspace{3mm} \= \hspace{3mm} \= \hspace{3mm} \= \hspace{3mm} \= %
\hspace{3mm} \= \hspace{3mm} \= \hspace{3mm} \= \hspace{3mm} \= %
\hspace{3mm} \= \hspace{3mm} \= \hspace{3mm} \= \hspace{3mm}  \kill
{\bf Input: } A DAG $G = (N, A)$ with dfs ordering, a path $P$,\\
~~~a family of paths $\mathcal{P}$ 
an integer $i \in [2, |N|]$.\\
%
{\bf Output: } Family  of paths in $G$ that can be extended from 
$P$.\\
$\mathcal{P}' := \emptyset$;\\
Let $Q_i :=$ set of dfs label of all children of the vertex with dfs label $i$;\\
{\bf if} $|Q_i| = 0$ {\bf then} $\mathcal{P}' := \mathcal{P}' \cup \{P\}$;\\
{\bf else}\+ \\
	{\bf while} $Q_i \neq \emptyset$ {\bf do} \+ \\
		Let $j$ be the smallest integer in $Q_i$;\\
		%$P: = \emptyset$; 
		%$P' := P$;\\
		Let $\y_i,$  and $\y_j$ be the labels of the vertices with 
		dfs label $i$ and $j$, respectively, and\\
		 ~~~the label of arc between $y_i$ and $\y_j$ is 
		 $(\w, m)$\\
		$P':= P \oplus ((\y_i, \y_j, \w, m))$;
		/* sequence concatenation */\\ 
		$\mathcal{P''}:=$ PathRecursion$(P', j, \mathcal{P}', G)$;
		$Q_i := Q_i \setminus \{j\}$;
		$\mathcal{P'}: = \mathcal{P'} \cup \mathcal{P''}$ \- \\		
	{\bf end while}\-\\
{\bf end if};\\
Output $\mathcal{P}'$ as the required family of extended paths.
\end{tabbing}

% % % % % % % % % % % % % % % % % % % % % % % % % % % % % % % % % % %  
% 
%             Complete Algorithm for three leaf 2-branches
% 
% % % % % % % % % % % % % % % % % % % % % % % % % % % % % % % % % % % 
\section{A Complete Algorithm to Compute Target $v$-components}

We briefly summarize how to use the procedures described thus far 
to obtain an algorithm.
Our global constants are 
vector $\x^* = (\x^*_{\intt}, \x^*_{\ex}, b)$, 
a non-negative integer $b$,
integer  
$\Delta_v \in [2, 3]$,  
element
$\ta_v \in \Lambda$,  \\
integers
$d_v \in [0, \val(\ta_v) - \Delta_v-1]$, 
$m_v \in [d_{v}, \val(\ta_v) - \Delta_v-1]$. \bigskip\\
%
\noindent {\sc CompleteAlgorithmVertex}(Global constants: 
$\ta_v, d_v, m_v, \Delta_v, \x^*_v$, ${\rm core~height}$)
\begin{tabbing}
%
% First, we set a tab stop every 3 mm
\hspace{3mm} \= \hspace{3mm} \= \hspace{3mm} \= \hspace{3mm} \= %
\hspace{3mm} \= \hspace{3mm} \= \hspace{3mm} \= \hspace{3mm} \= %
\hspace{3mm} \= \hspace{3mm} \= \hspace{3mm} \= \hspace{3mm}  \kill
% Throughout the description we use \+ and \- to advance and reduce by one tab stop
% the indent of the following lines, and by necessity, use \> to advance the indent 
% of the current line
%
% the collection  
                

Let $\ell:= |\Gamma^{\inn}| + 2$;\\
%Let $\delta_1 := \lfloor (\ell - 2-1)/2 \rfloor$,  
%$\delta_2 := \lceil (\ell - 2-1)/2 \rceil$ ;\\
%{\bf for each} base-vertex $v \in V_B$ {\do} \+ \\
$t : = {\rm core~height} -3$;\\
	Compute $\V^{(0)}_{\co + \Delta_v}
	(\ta_v, d_v, m_v, h; \x^*_v)$ for a fixed 
	$(\ta_v, d_v, m_v, \Delta_v)$, \\
	~~and for each 
	$h \leq \ell  $ if 
	$\ell \leq 2$ and $\x^*_v(\tt bc)$ = 0;\\
	%
	Compute $\V^{(0)}_{\co + (\Delta_v + 1)}
	(\ta_v, d_v, m, h; \x^*_v)$ for a fixed 
	$(\ta_v, d_v, \Delta_v)$, \\
	~~for each $m \in [d_v -1, \val(\ta_v) - \Delta_v -1]$, 
	$h \leq 2$ if 
	$\ell > 2$ and $\x^*_v(\tt bc)$ = 1;\\
	%
	Compute 
	$\V_{\en}^{(0)}(\ta, d, m; \x^*_v)$ for each 
$\ta \in \Lambda$, $d \in [1, \val(\ta)-1]$, \\
~~$m \in [d, \val(\ta)-1]$ if 
	$\ell > 2$ and $\x^*_v(\tt bc)$ = 1;\\
	%
	Compute $\V_{\inl}^{(0)}(\ta, d, m; \x^*_v)$ for each 
	$\ta \in \Lambda$, $d \in [0, \val(\ta)-2]$, \\
~~$m \in [d, \val(\ta)-2]$
	if 
	$\ell > 2$ and $\x^*_v(\tt bc)$ = 1;\\
	%
	Compute 
	$\V_{\en}^{(h)}(\tb, d', m'; \x^*_v)$ for each 
$\tb \in \Lambda$, $d' \in [1, \val(\tb)-1]$, \\
~~$m' \in [d', \val(\tb)-1]$, $1 \leq h \leq t$, 
     if 
	$\ell > 2$ and $\x^*_v(\tt bc)$ = 1;\\
	%	
%	Compute $\V_{\co+(\Delta_v + 1)}^{(0)}(\ta', d_{\ta'}, m_{\ta'}; \x_v^*)$ , for  
%$\ta' \in \Lambda$, \\
%~~integers
%$d_{\ta'} \in [1, \val(\ta') - 1]$, \\
%~~$m_{\ta'} \in [d_{\ta'}, \val(\ta')-1]$, if $\ell > 2$ and $\x^*_v(\tt bc)$ = 1;\\
%Compute the set FP of feasible pairs 
%$(\y, \y')$ such that $\y  +\y' + \1_{\gamma} = \x^*$;\\
Compute the DAG $G$ representation of $\x^*_v$;\\
Enumerate the set $\mathcal{P}$ of paths 
from sources to leaves in $G$;\\
{\bf for each} path $P$ in $G$ {\bf do}\+ \\
 Let $P := ((\x^*, \y_h, \w_h, m_h), (\y_h, \y_{h-1}, \w_{h-1}, m_{h-1}), \ldots, (\y_1, \y_{0}, \w_{0}, m_{0}))$;\\
~~where 
$\w_h \in \V_{\co+(\Delta_v + 1)}^{(\delta_1)}$, 
$\w_{h-1}, \ldots, \w_{1}\in 
\V_{\inl}^{(0)}$,
$\w'_0 \in \V_{\en}^{(0)} $, $h = t$;\\
Get a target $v$-component by using the trees 
corresponding to\\
~~ $\w_h, \w_{h-1}, \ldots, \w_0 $\\
Get the number of $v$-components obtained by the path 
$P$ \\
~~ $n(\w_h)\times \cdots \times n(\w_0)$, where 
$n(\w_h), \ldots, n(\w_0),$ 
are the number of trees with vector \\
~~ $\w_h, \ldots, \w_0$, respectively\- \\
{\bf end for}.
\end{tabbing}



% \newpage
% % % % % % % % % % % % % % % % % % % % % % % % % % % % % % % % % % % % % % % % % 
% 
%                 Getting Vectors of bi-rooted core-subtrees
% 
% % % % % % % % % % % % % % % % % % % % % % % % % % % % % % % % % % % % % % % % % 
\section{Generation of Frequency Vectors of Bi-rooted Core-subtrees}
For an integer $h \in [h_1, h_2]$, elements
$\ta, \ta^{e} \in \Lambda$, integers
$d  \in [1, \val(\ta) - 1]$, 
$m\in [d,\val(\ta)-1]$, 
$\Delta^e \in [1, \val(\ta^e) -1] $, 
$m^e \leq \val(\ta^e) - \Delta^e$,
and $q \geq 1$, 
 we give a procedure to 
compute the set $\V_{\co+1, \Delta^e}^{(q)}(\ta, d, m, \ta^e, 1, m^e, h; \x^*)$.
\bigskip\\
\noindent {\sc ComputeBiRootedCoreSubtree($\ta, d, m, \ta^e, 1, m^e, h, q$)}
\begin{tabbing}
%
% First, we set a tab stop every 3 mm
\hspace{3mm} \= \hspace{3mm} \= \hspace{3mm} \= \hspace{3mm} \= %
\hspace{3mm} \= \hspace{3mm} \= \hspace{3mm} \= \hspace{3mm} \= %
\hspace{3mm} \= \hspace{3mm} \= \hspace{3mm} \= \hspace{3mm}  \kill
% Throughout the description we use \+ and \- to advance and reduce by one tab stop
% the indent of the following lines, and by necessity, use \> to advance the indent 
% of the current line
%

%%% Important %% In the next 2 can be replaced by \rho to 
%%% make the algorithm general
{\bf Input:}
An integer $h\geq 0$, elements  
$\ta, \ta^{e} \in \Lambda$, integers
$d  \in [1, \val(\ta) - 1]$, 
$m\in [d,\val(\ta)-1]$,  \+ \\
$\Delta^e \in [1, \val(\ta^e) -1] $, 
$m^e \leq \val(\ta^e) - \Delta^e$,
and $q \geq 1$.\- \\
\hspace{3mm}  /* Global data: A vector $\x^* = (\x^*_{\intt}, \x^*_{\ex}, b)$, \+ \\
a non-negative integer $b$,
                 the collection \\
                 $\Vv_{\co+ 2}^{(0)}$  vector sets 
                 $\V_{\co + 2}^{(0)}(\ta, d-1, m_{\ta}, p; 
                 \x^*)$, 
                 %$d_{\ta} \in [0, \val(\ta)-3]$,
                 $m_{\ta} \in [d-1, \val(\ta) -\Delta -1]$, 
                 $p \in [0,h]$,\\
                 for $q \geq 2$, 
                 $\Vv_{\en}^{(q-1)}$ of vector sets  
                 $\V_{\co+1, \Delta^e}^{(q-1)}
                 (\tb, d', m', \ta^e, 1, m^e, h'; \x^*)$, \\
                 $\tb \in \Lambda$,
                  $d' \in [1, \val(\tb)-1]$, 
                  $m' \in [d', \val(\tb) - 1]$, 
                  $h' \in [0, h]$, integer $g \geq 1$. */ \- \\
{\bf Output:} The set $\V_{\co+1, \Delta^e}^{(q)}(\ta, d, m, \ta^e, 1, m^e, h; \x^*)$, where 
we store these vectors in a trie. \\
$\W := \emptyset$; \\
{\bf for each} triplet $(\ta, d-1, m_{\ta}, p)$ {\bf do} \+ \\
	{\bf if} $q = 1$ {\bf then}\+ \\
		{\bf if} $p = h$ and 
	     $\val(\ta) \geq m_{\ta} + m^e$ {\bf then} \+ \\
	     	{\bf for each} $\w^{\ta}  \in 
                      \V_{\co +2}^{(0)}(\ta, d-1,m_{\ta}, p; \x^*)$ 
                      {\bf do} \+\\                
				$\gamma^{\intt} := (\ta d, \ta^e 1, m^e )$; 
				$\y := \y^{\ta} + \1_{\gamma^{\intt}}$\\
				  {\bf if}  $\gamma^{\intt} \in \Gamma^{\intt}$ and 
						$\y \leq \x^*$ {\bf then}\+ \\
						                     {\bf if} $\y \in 
                     \V$ {\bf then} \+ \\
                              
                     $\V := \V \cup \{ \y \}$ \- \\
				{\bf end if} \- \\
			{\bf end if} \- \\
			{\bf end for} \- \\
		{\bf end if} \- \\
	{\bf else} /* $q>1$ */ \+ \\
		{\bf for each} triplet  $(\tb, d_{\tb}, m_{\tb}, h')$ {\bf do} \+ \\
		  {\bf for each} $\y^{\tb} \in 
   \V_{\co+1, \Delta^e}^{(q-1)}
                 (\tb,d_{\tb}, m_{\tb}, \ta^e, 1, m^e, h'; \x^*)$
                         {\bf do} \+ \\
            {\bf for each} $m' \in [1, 3]$ such that  \\ 
                      ~~~- $\gamma^{\intt} := (\ta d, \tb  \{d_{\tb} + 1\},
                       m') \in \Gamma^{\intt}$ and\\
                      ~~~- $m_{\ta} + m' = m, 
                      m_{\ta} + m' + 1 \leq \val(\ta)$, 
                      $m' +m_{\tb} \leq \val(\tb)$, \\
                     ~~~- $h = \max\{p, h'\}$ and \\
                      ~~~- $\y:= \y_{\ta} + \y_{\tb} + 
                      \1_{\gamma^{\intt}} \leq \x^*$
                                             {\bf do} \+ \\
                                      {\bf if} $\y \in 
                     \V$ {\bf then} \+ \\                         
                     $\V := \V \cup \{ \y \}$; \- \\
                      {\bf end if} \- \\  
		{\bf end for} \- \\  
		{\bf end for} \- \\
		{\bf end for} \- \\
	{\bf end if} \- \\
{\bf end for}; \\
Output $\W$ as 
$\V_{\co+1, \Delta^e}^{(q)}(\ta, d, m, \ta^e, 1, m^e, h; \x^*)$.
\end{tabbing}
% \newpage
% % % % % % % % % % % % % % % % % % % % % % % % % % % % % % % % % % % % % % % % % % 
% 
%     Computing DAG
% 
% % % % % % % % % % % % % % % % % % % % % % % % % % % % % % % % % % % % % % % % % % 
\section{Computing DAG Representation for $e$-Components}

%For an integer $h \geq 1$, element 
%$\ta \in \Lambda$,
%integers $d  \in [1,\val(\ta)-1]$, and 
%$m\in [d,\val(\ta)-1]$ 
% we give a procedure to 
%compute the set $\V_{\en}^{(h)}(\ta, d, m; \x^*)$.
%\bigskip\\
\noindent {\sc DAGRepresentationEdge($\ta^e_i, m^e_i, \Delta^e_i,  \delta_i, h_i\x^*$)}
\begin{tabbing}
%
% First, we set a tab stop every 3 mm
\hspace{3mm} \= \hspace{3mm} \= \hspace{3mm} \= \hspace{3mm} \= %
\hspace{3mm} \= \hspace{3mm} \= \hspace{3mm} \= \hspace{3mm} \= %
\hspace{3mm} \= \hspace{3mm} \= \hspace{3mm} \= \hspace{3mm}  \kill
%
{\bf Input:}
%Element $\ta \in \Lambda$, 
%integer $d \in[1,\val(a)-1]$, 
%$m \in [d, \val(a)-1]$, $h \geq 1$. \\
\hspace{3mm}  /* Global data: A vector $\x^* = (\x^*_{\intt}, \x^*_{\ex}, b)$, \+ \\
a non-negative integer $b$,\\ 
$\ta_i^{e} \in \Lambda$, integers
$\Delta_i^e \in [1, \val(\ta_i^e) -1] $, 
$m_i^e \leq \val(\ta_i^e) - \Delta_i^e$,\\
                 the collection  
                 $\Vv^{(0)}_{\inl}$ vector sets 
                 $\V^{(0)}_{\inl}(\ta, d, m; \x^*)$, \\
                 %$d_{\ta} \in [0, \val(\ta)-3]$,
                 %$m_{\ta} \in [d-1, \val(\ta) -2]$\\
                 integers $\delta_i\geq 0$, $h_i \geq 1$, $i = 1, 2$, \\
                 $\Vv^{(h)}_{\en}$ of vector sets 
                 $\V_{\en}^{(h)}(\ta_1, d_1, m_1; \x^*)$, 
                 $\ta_1 \in \Lambda$,
                  $d_1 \in [1, \val(\ta_1)-1]$, \\
                  $m_1 \in [d_1, \val(\ta_1) - 1]$, 
                  $1 \leq h \leq \max\{\delta_1, \delta_2\}$, \\
                  the collection 
                   $\Vv_{\en}^{(0)}$ of sets 
                  $\V_{\en}^{(0)}(\ta_1, d_1, m_1; \x^*)$, 
                  $\ta_1 \in \Lambda$,
                  $d_1 \in [1, \val(\ta_1)-1]$, \\
                  $m_1 \in [d_1, \val(\ta_1) - 1]$,\\
                  the collection $\Vv_{\co+2}^{(0)}$ of sets 
                 $\V_{\co+2}^{(0)}(\ta, d , 
                 m, h; \x^*)$
                  for all possible $\ta, d, m$ and 
                  $h \leq \max\{h_1, h_2\}$,\\
                  $\Vv_{\co+(\Delta + 1)}^{(0)}$ of sets 
                 $\V_{\co+(\Delta + 1)}^{(0)}(\ta, d -1, 
                 m'', p; \x^*)$
                  for $p \leq 2$,\\
                  for $2 \leq q_i \leq \delta_i, i=1, 2$, 
                 families $\Vv_{\en, i}^{(q_i)}(\ta^e_i, m^e_i)$ 
                 of vector sets  
                 $\V_{\co+1, \Delta_i^e}^{(q_i)}
                (\ta_i,d_i, m_i, \ta_i^e, 1,m_i^e, h_i; \x^* )$.             
                  */ \- \\
{\bf Output:} A set of feasible pairs $\y_i, i=1, 2$ of length 
$\delta_i, i = 1, 2$, respectively, \+ \\
two vertex-labeled and edge-labeled DAG representation of 
these feasible pairs of e-component, \\
and DAG representations of 
frequency vector of each non-core part of the e-component\\ 
with frequency vector $\x^*$. \- \\
%
$F:= \emptyset$; /* to store feasible pairs for core part */\\
$G_i := (N_i, A_i)$; $A_i:= \emptyset$; 
$N_i:= \emptyset, i = 1, 2;$/*  core part */\\
%{\bf if } $\delta_1 + \delta_2 + 1 \geq 2$ {\bf then} \+ \\%1
{\bf for each} $(\ta_i,d_i, m_i), i = 1, 2$\+ \\
{\bf for each} $\gamma=(\ta_1\{d_1+1\}, \ta_2\{d_2+1\}, m) \in \Gamma^{\intt}$
 with \\
 ~~~~~$m \in [1, \min\{3, \val(\ta_1)-m_1, \val(\ta_2)-m_2\}]$   {\bf do} \+ \\
 Let
   $L_1$ denote the sorted  list of vectors in 
   $\V_{\co+1, \Delta_1^e}^{(\delta_1)}
                (\ta_1,d_1, m_1, \ta_1^e, 1,m_1^e, h_1; \x^* )$; \\
 Construct the set $\overline{\W}:=
 \{\overline{\z} \mid \z \in \V_{\co+1, \Delta_2^e}^{(\delta_2)}
                (\ta_2,d_2, m_2, \ta_2^e, 1,m_2^e, h_2; \x^* )\}$
 of  the $\gamma$-complement vectors; \\ 
 Sort the vectors in  $\overline{\W}$ to
obtain a sorted list $L_2$;\\
Merge $L_1$ and $L_2$ into a single sorted list 
$L_{\gamma}$
of vectors in both lists (as a multiset);\\
Trace the list  $L_{\gamma}$ and for each consecutive pair 
$\z^1, \z^2$ of vectors 
       with $\z^1 = \z^2$\\
$\y_1: = \z^1, \y_2: = \overline{\z^2}$ is a feasible pair;\\
$N_i: = N_i \cup \{ (\y_i, \delta_i;\ta_i, d_{i}, m_{i}, h_i )$;
 		\\
 		%$A:= A \cup \{a_{\y_1\y_2}\}$ and\\
 	 Let the label of the arc from 
 		$\y_1$ to $\y_2$ is $(\0, m)$; \\
 		$F: = F \cup \{(\y_1, \y_2; \0, m';  
 		\ta_1, d_1, m_1, h_1; \ta_2, d_{2}, m_{2}, h_2)\}$\-  \\
 {\bf end for} \- \\% gamma
 {\bf end for}; \\%triplet   
$\mathcal{C}:= \emptyset$; \\/* a set of vectors of rooted core subtrees for which we calculate DAG in second phase */	\\	

	$G' := G_2$;\\
	{\bf for each} $\ell \in (\delta_2, \ldots, 1)$ {\bf do}\+ \\ % 4
		$(G'':=(N'', A''), \mathcal{D}) :=$ 
		CoreDAGSublayer($\Vv_{\co+1, 2}^{(\ell-1)}, G',
		 \ell-1, \Vv_{\co+2}^{(0)}, h_2$);\\
		~~~$N_2:= N_2 \cup N''; A_2:= A_2 \cup A''$; 
		$\mathcal{C}:= \mathcal{C} \cup \mathcal{D}$\- \\		
	{\bf end for};\\
	%
	$G' := G_1$;\\
	{\bf for each} $\ell \in (\delta_1, \ldots, 1)$ {\bf do}\+ \\ % 4
		$(G'':=(N'', A''), \mathcal{D}) :=$ 
		CoreDAGSublayer($\Vv_{\co+1, 1}^{(\ell-1)}, G',
		 \ell-1, \Vv_{\co+2}^{(0)}, h_1$);\\
		~~~$N_1:= N_1 \cup N''; A_1:= A_1 \cup A''$; 
		$\mathcal{C}:= \mathcal{C} \cup \mathcal{D}$\- \\		
	{\bf end for};\\
	{\bf for each} $(\y, \ta, d, m, t ) \in \mathcal{C}$ {\bf do}\+ \\
	$G''' := (N''', A''')$; $N''':= \{\y\}; A''':= \emptyset$;\\
	{\bf for each} $\ell \in (t-2, \ldots, 1)$ {\bf do}\+ \\ % 4
	{\bf if} $\ell = t-2$ {\bf then} \+ \\
		$G^*:=(N^*, A^*) :=$ 
		DAGSublayer($\Vv_{\co + 2}^{(0)}(\ell - 1, \y), G''',
		 \ell-1,\Vv_{\co+\Delta + 1}^{(0)}(\ta, d, m; \y)$),\\
		 ~~where $\Vv_{\co + 2}^{(0)}(\ell - 1, \y)$ 
		 is a family of 
		 vectors of end-subtrees under $\y$\\
		 ~~  with core height 
		 $\ell - 1$, 	\\	 
		~~ $\Vv_{\co+\Delta + 1}^{(0)}(\ta, d, m; \y)$ is the 
		 family of sets $\V_{\co+\Delta + 1}^{(0)}(\ta, d, m, p; \y)$;\\
		~~~$N''':= N''' \cup N^*; A''':= A''' \cup A^*$;	\- \\	
		%%%%%%
		{\bf else} /* $\ell < t-2 $ */ \+\\
			$G^*:=(N^*, A^*) :=$ 
			DAGSublayer($\Vv_{\co + 2}^{(0)}(\ell - 1, \y), G''',
		 \ell-1,\Vv_{\inl}$),\\
		 ~~ where $\Vv_{\co + 2}^{(0)}(\ell - 1, \y)$ 
		 is a family of 
		 vectors of end-subtrees under $\y$ \\
		 ~~ with core height $\ell - 1$;\\
		~~~$N''':= N''' \cup N^*; A''':= A''' \cup A^*$;	\- \\	
		{\bf end if}\-\\
	{\bf end for}; \\
	Output $(\y, G'''')$\- \\
	{\bf end for};\\
Output $G_i, i= 1, 2$ as DAG representations and 
the set $F$.
\end{tabbing}

\noindent {\sc CoreDAGSublayer($\Vv, G, \ell, \Vv', h$)}
\begin{tabbing}
%
% First, we set a tab stop every 3 mm
\hspace{3mm} \= \hspace{3mm} \= \hspace{3mm} \= \hspace{3mm} \= %
\hspace{3mm} \= \hspace{3mm} \= \hspace{3mm} \= \hspace{3mm} \= %
\hspace{3mm} \= \hspace{3mm} \= \hspace{3mm} \= \hspace{3mm}  \kill
{\bf Input:} 
A family $\Vv'$ of vectors rooted core-subtrees with a root label $\ta_1$, \+ \\
degree 
$d_1$ and multiplicity $m_1$ and core height $t \leq h$, 
$G = (N, A)$,  \\
a family   
                 $\Vv$ of vectors of bi-rooted core subtrees with core height at most $h$ and         \\
                 $\ell$ (the height of the layer  
                 that we add in $G$ at this stage).\- \\
%                 
{\bf Output:}  A DAG $G'$ that is a super-graph of $G$, and 
a set of vectors of rooted core subtrees.\\
%
$G':= G$; $\mathcal{C}:= \emptyset;$\\
{\bf for each} $\y_1 \in \Vv$ {\bf do} \+ \\%1
	{\bf for each} $\y_1' \in \Vv'$ {\bf do} \+ \\%2
	 Let the height of $\y_1$  and $\y'_1$ be $t$ and $t'$,
	  respectively;\\
		{\bf if } there exists $\gamma \in \Gamma^{\inn}$ and 
		some $\y_2 \in N$ such that \\
		~~~$\y_i, i=1,2$ are feasible, i.e., 
		$\y_1 +\y'_1+1_{\gamma} = \y_2$ and 
		$\max\{t, t'\} = h$ {\bf then}\+\\%4
%		{\bf for each} $\gamma=(\ta\{d+2\}, \ta_1\{d_{1}+1\}, m'')
%                  \in \Gamma^{\inn}$ with \\
%                 ~~~~~$m'' \in [1, \min\{3, \val(\ta)-m, \val(\tb)-m_{\tb}\}]$ {\bf do}\+ \\ %3
%		 {\bf if} there exists  
				{\bf if} $\y_1 \not\in N$ {\bf then} 
				$N:= N \cup \{(\y_1, \ell;  \ta_1, 
				d_1, m_1, t')\}$;\\
				$A:= A \cup \{a_{\y_2\y_1}\}$ and \\
			~~~label the arc from 
				$\y_2$ to $\y_1$ by $(\y', m)$, \\
				~~~where $m$ is 
				the bond multiplicity in $\gamma$; \\
				$\mathcal{C}:= \mathcal{C} \cup
				\{(\y'_1, \ta_1, d_1, m_1, t' )\}$\- \\
               {\bf end if} \-\\%4    
		%	{\bf end for} \- \\%3
	{\bf end for} \- \\%2
{\bf end for};  \\%1
Output $G'$ as a required DAG and $\mathcal{C}$ the required set of rooted core subtrees.
\end{tabbing}

%%%
%%%
%%%
%%%
\newpage
\section{A Complete Algorithm to Compute Target $e$-components}

We briefly summarize how to use the procedures described thus far 
to obtain an algorithm.
Our global constants are a frequency vector $\x^*_{\rm e}$ of an e-component, 
two fixed tuples $(\ta^e_j, m^e_j, \Delta^e_j), j = 1, 2$
a lower bound 
${\rm ch}_{{\rm LB}}(e)$ and an upper bound
${\rm ch}_{{\rm UB}}(e)$ on core height, where we take 
$\rho = 2$. \bigskip\\
%
\noindent {\sc CompleteAlgorithmEdge}(Global constants: 
$\ta^e_j, m^e_j, \Delta^e_j, \x^*_e, {\rm core~height~bounds}$)
\begin{tabbing}
%
% First, we set a tab stop every 3 mm
\hspace{3mm} \= \hspace{3mm} \= \hspace{3mm} \= \hspace{3mm} \= %
\hspace{3mm} \= \hspace{3mm} \= \hspace{3mm} \= \hspace{3mm} \= %
\hspace{3mm} \= \hspace{3mm} \= \hspace{3mm} \= \hspace{3mm}  \kill
% Throughout the description we use \+ and \- to advance and reduce by one tab stop
% the indent of the following lines, and by necessity, use \> to advance the indent 
% of the current line
%{\bf for each} base-edge $e \in E_B$ {\do} \+ \\
$\Gamma^{\inn}_e: = \text{The set internal edges in 
	$\x^*_e$}$;\\
Compute $\V^{(0)}_{\co + (\Delta+1)}
	(\ta, d, m, h; \x_e^*)$ for each \\
		~~
		$\Delta \in [2,3]$, 
		$\ta \in \Lambda$, $d \in [0, \val(\ta)-\Delta]$,
		$m \in [d, \val(\ta) -\Delta]$, 
	$h \in [0, 	\min \{2, {\rm ch}_{{\rm UB}}(e)  \} ]$;\\
	%
	Compute 
	$\V_{\en}^{(0)}(\ta, d, m; \x^*_e)$ for each 
	$\ta \in \Lambda$, $d \in [1, \val(\ta)-1]$,
 	$m \in [d, \val(\ta)-1]$;\\
	%
	Compute $\V_{\inl}^{(0)}(\ta, d, m; \x^*_e)$ for each 
	$\ta \in \Lambda$, $d \in [0, \val(\ta)-2]$, 
	$m \in [d, \val(\ta)-2]$;\\
	%
	
	Compute 
	$\V_{\en}^{(h)}(\ta, d, m; \x^*_e)$ for each 
	$\ta \in \Lambda$, $d \in [1, \val(\ta)-1]$, \\
~~$m \in [d, \val(\ta)-1]$, 
     $1 \leq h \leq \min\{|\Gamma^{\inn}_e| -1, 
     {\rm ch}_{{\rm UB}}(e)-2-1\}$
     if 
	${\rm ch}_{{\rm UB}}(e) > 2$;\\
	%
	Compute 
	$\V_{\co + \Delta}^{(0)}(\ta, d, m, h; \x^*_v)$ for each 
     $\Delta \in [2, 3]$, 
     $\ta \in \Lambda$, $d \in [1, \val(\ta)-1]$, \\
~~$m \in [d, \val(\ta)-1]$, 
     $h \leq \min\{|\Gamma^{\inn}_e| +2, 
     {\rm ch}_{{\rm UB}}(e)\}$, 
     if 
	${\rm ch}_{{\rm UB}}(e) > 2$ ;\\
	%
	Compute $\V_{\co+1, \Delta^e_j}^{(q)}(\ta, d, m, \ta^e_j, 1,
	 m^e_j, h; \x^*_e)$ for fixed 
	$(\ta^e_j, m^e_j, \Delta^e_j)$, 
	$\ta, \in \Lambda$, \\
	~~integers
	$d  \in [1, \val(\ta) - 1]$, 
	$m\in [d,\val(\ta)-1]$, 
	$q = \Delta_j ^e, j = 1, 2$;\\
	%
%	{\bf for each} two tuples $(\ta_j, d_j, m_j, \ta^e_j, 1,
%	 m^e_j, h; \x^*_e)$, $j = 1, 2$ {\bf do} \+ \\
%      search for a feasible vector pair
%            in the pair of sets 
%            $\W_{\co+1, \Delta^e_j}^{(q)}(\ta_j, d_j, m_j, 
%            \ta^e_j, 1,  m^e_j, h_j; \x^*_e)$  \- \\
%            
%{\bf end for}.\\
Compute the set FP of feasible pairs 
$(\z, \z')$ such that $\z  +\z' + \1_{\gamma} = \x_e^*$;\\
Compute the DAG $G_1$ (resp., $G_2$) representation of 
all vectors $\z$ (resp., $\z'$)\\
~~ such that $(\z, \z') \in {\rm FG}$ 
(resp., $\z' \in {\rm FG}$);\\
Enumerate the set $\mathcal{P}_1$ (resp., $\mathcal{P}_2$) of paths
from sources to sinks in $G_1$ (resp., $G_2$);\\
{\bf for each } feasible pair $(\z, \z') \in {\rm FG}$ {\bf do}\+ \\
Let $P := ((\z, \z_h, \y_h, m_h), (\z_h, \z_{h-1}, \y_{h-1}, m_{h-1}), \ldots, (\z_1, \z_{0}, \y_{0}, m_{0}))$;\\
$P' := ((\z', \z'_{h'}, \y'_{h'}, m'_{h'}), (\z'_{h'}, \z'_{h'-1}, \y'_{h'-1}, m'_{h'-1}), \ldots, (\z'_1, \z'_{0}, \y'_{0}, m'_{0}))$, \\
%~~where 
%$\w_h \in \V_{\co+(\Delta_v + 1)}^{(\delta_1)}$, 
%$\w_{h-1}, \ldots, \w_0, \w'_{h'}, \ldots, \w'_{1}\in 
%\V_{\inl}^{(0)}$,
%$\w'_0 \in \V_{\en}^{(0)} $;\\
Compute DAG representation $G^i$ (resp., $(G')^{i}$ of each 
$\y_i$ (resp., $\y'_i$). \\
%Let $n(\y_i)$ (resp., $n(\y'_i)$) denote the the number of 
%graphs that can be obtained from $\y_i$ (resp., $\y'_i$)\\
% as explained in {\sc CompleteAlgorithmVertex}. \\
Get a target $e$-component by using the trees corresponding to\\
~~ $\y_h, \y_{h-1}, \ldots, \y_0, \y'_{h'}, \ldots, \y'_{0}$ \\
Get the number of target $e$-components obtained by paths 
$P$ and $P'$ as \\
~~ $(n(\y_h)\times \cdots \times n(\y_0)) \times 
(n(\y'_{h'})\times \cdots \times n(\y'_{0}))$, \\
~~ where 
$n(\y_i)$ (resp., $n(\y'_i)$) denote the number of 
graphs that can be obtained\\
~~ from $\y_i$ (resp., $\y'_i$)
as explained in {\sc CompleteAlgorithmVertex}\- \\
{\bf end for}.

\end{tabbing}
%%%%%%%%%%%%%%%%%%%%%%%%%%%%%%%%%%%%%%%%%%%%%%%

\section {Canonical Representation of Fringe Trees}

For a graph $G$, let $V(G)$ and $E(G)$ denote the vertex set and edge set of $G$, respectively. 
We denote by $(u, v)$ a directed edge from vertex $u$ to vertex $v$ in 
a graph. 
However, we denote by $uv$ an undirected edge between $u$ and $v$ in a graph, where we assume that $uv =vu$. 
For a vertex $v$, we denote by $N_G(v)$ neighbors of $v$. 
%For a vertex and edge colored graph $G$, a vertex $v \in V(G)$ and 
%$e \in E(G)$, let col$(v)$ and col$(e)$ denote the 
%color of $v$ and $e$, respectively. 
For an edge weighted graph $(G, w)$ with weight function $w$ and an edge 
$e \in E(G)$, we denote by $w(e)$ the weight of the edge  $e$. 
For a rooted tree $T$ and a non-root vertex $v \in V(T)$, we denote by ${\rm prt}_T(v)$ the parent of $v$ in $T$. 
For a rooted tree $T$ and a vertex $v \in V(T)$, we denote by ${\rm d}_T(v)$ 
depth of $v$ in $T$, i.e., the length of the path between $v$ and the root of $T$. 
When the underlaying tree $T$ is fixed, then we simply denote parent and 
depth by ${\rm prt}(v)$ and~${\rm d}(v)$. 
 
 
 Let $(T,w, \lambda)$ be a tree with $n$ vertices, 
 rooted at $r$ with weight function
 $w : E(T) \to \{1, 2, 3\}$, a  coloring function 
 $\lambda : V(T) \to \{1, 2, \ldots, k\}$ for some 
 $k$. 
Let $K= (T, \pi)$ be an ordered tree of $T$ with a left-to-right ordering $\pi$ 
on the children of each vertex, and 
$v_1, v_2, \ldots, v_n$ indexing on the vertices of $H$ obtained by 
depth-first-search starting from the root $r$ and 
visiting children following $\pi$. 
We define, a (color, depth)-sequence $\psi(K)$ to be the sequence
\[
\psi(K) \triangleq (( \lambda(v_1), \d(v_1)), (\lambda(v_2), \d(v_2)), \ldots, (\lambda(v_n), \d(v_n)))), 
\]
a weight-sequence $\sigma(K)$ to be the sequence 
$$
\sigma(K) \triangleq (w_2, w_3,\ldots, w_n), 
$$ where 
$w_i = w(v_i\prt(v_i))$ for $i \in [2, n]$.
We define a canonical representation C$(T)$ of $T$ to the pair 
$(\psi(K), \sigma(K))$ such that $\psi(K)$ is lexicographically maximum among  
the (color, depth)-sequence of all ordered trees of $T$ and 
$\sigma(K)$ is lexicographically maximum among the 
weight- sequence of all ordered trees of $T$. 
In such a case, we call $K$ the left-heavy representation of $T$. 
For a vertex $v \in T,$ we denote by $T\langle v \rangle$ the subtree of $T$ rooted at $v$ that consists of $v$ and all its 
descendants. 

For two sequences $S$ and $S'$, we denote by 
$S \oplus  S'$ the concatenation of $S$ with $S'$. 
 
We present a procedure to compute the frequencies of 2-fringes in 
a given set $\mathcal{G}$ of chemical graphs, where 
for a vertex $v$, we use atomic number of an atom as color 
$\lambda(v)$ and 
$w$ as the multiplicity between the edges. \newpage
%%%%%
\noindent 
{\bf Algorithm}{~\sc CompFreq ($\mathcal{G}$)}
\begin{tabbing}
%
% First, we set a tab stop every 3 mm
\hspace{3mm} \= \hspace{3mm} \= \hspace{3mm} \= \hspace{3mm} \= %
\hspace{3mm} \= \hspace{3mm} \= \hspace{3mm} \= \hspace{3mm} \= %
\hspace{3mm} \= \hspace{3mm} \= \hspace{3mm} \= \hspace{3mm}  \kill
% Throughout the description we use \+ and \- to advance and reduce by one tab stop
% the indent of the following lines, and by necessity, use \> to advance the indent 
% of the current line
{\bf Input: } A set of chemical graphs $\mathcal{G}$.\\
{\bf Output: } Frequencies of 2-fringe trees in $\mathcal{G}$.\\
Let $\mathcal{C}:= \emptyset;$ 
/* canonical representation of distinct 2-fringe trees */\\
%
{\bf for each} $G \in \mathcal{G}$ {\bf do}\+\\
	Let $G' := G$; Remove all leaves from $G'$ in two rounds to get roots of 2-fringe trees;\\
	{\bf for each} $v\in V(G')$ such that 
	     $N_G(v) \setminus N_{G'}(v) \neq \emptyset$ {\bf do}\+ \\
	      Let $(T, w, \lambda)$ be the tree rooted at $v$
	      obtained by performing dfs 
	      from $v$ to \\
	      ~~its descendants and $T$ 
	      satisfies the degree condition of
	      2-fringe trees; \\
	      ~~/* $|V(T)| \leq 2d +2$ where $d$ is the
	      number of children of $v$ /*\\
			C[T] := {~\sc CanonRecur($T, v$)}\\
			{\bf if} $C[T] \not\in \mathcal{C}$ {\bf then} 
					${f_{C[T]} : = 1}$\\
			{\bf else} $f_{C[T]} : = f_{C[T]} + 1$ \\
			{\bf endif};\\
			 $\mathcal{C} := \mathcal{C} \cup \{C[T]\}$; \- \\
	{\bf endfor}\- \\ % v
{\bf endfor}\\ % \G
Output $C[T]$ and  $f_{C[T]}$ as the canonical representation and\\
~~frequency of fringe tree $T$, respectively, for each 
$C[T] \in \mathcal{C}$.
\end{tabbing}\bigskip
%%%
{\bf Algorithm}{{~\sc CanonRecur($T, v$)}
\begin{tabbing}
%
% First, we set a tab stop every 3 mm
\hspace{3mm} \= \hspace{3mm} \= \hspace{3mm} \= \hspace{3mm} \= %
\hspace{3mm} \= \hspace{3mm} \= \hspace{3mm} \= \hspace{3mm} \= %
\hspace{3mm} \= \hspace{3mm} \= \hspace{3mm} \= \hspace{3mm}  \kill
{\bf Input: } A vertex colored and edge weighted tree rooted 
tree $(T, w, \lambda)$ and a vertex $v$.\\
{\bf Output: } The canonical representation 
$C(T\langle v \rangle)$.\\
{\bf if} $v$ is a leaf {\bf then} 
	%%{\bf if}  $|V(T)| = 1$ {\bf then} 
  %%$C[T\langle v \rangle]:= (\lambda(v), 0)$\\
   $C[T\langle v \rangle] := 
  		(\lambda(v), \d(v))$ 
  {\bf endif}\\
%
{\bf else}\+ \\
	{\bf for each} child $u$ of $v$ {\bf do} \+ \\
		$C[T\langle u \rangle] := (\psi[T\langle u \rangle], 
		\sigma[T\langle u \rangle]) := $
		{\sc CanonRecur($T, u$)}\- \\
	{\bf endfor}	\\
	Let $v_1, v_2, \ldots v_n$ be the indexing of children of 
	$v$ such that for each $i \in [1, n-1]$,\\
	~~it holds that 
	$(\psi[T\langle v_i \rangle], w(v_iv) \oplus
		\sigma[T\langle v_i \rangle])$ is lexicographically 
		larger or equal to \\
		~~$(\psi[T\langle v_{i+1} \rangle], w(v_{i+1}v) \oplus
		\sigma[T\langle v_{i+1} \rangle])$\\
		Let $\psi[T\langle v \rangle]:= 
		(\lambda(v), \d(v)) \oplus
		\psi[T\langle v_1 \rangle]\oplus \cdots \oplus 
		((\lambda(v_n), \d(v_n))) \oplus
		\psi[T\langle v_n \rangle]$ and \\
		~~$\sigma[T\langle v \rangle]:= 
		w(v_{1}v) \oplus
		\sigma[T\langle v_1 \rangle]\oplus \cdots \oplus 
		w(v_{n}v) \oplus
		\sigma[T\langle v_n \rangle]$;
		$C[T\langle v \rangle] := (\psi[T\langle v \rangle], 
		\sigma[T\langle v \rangle])$	\- \\
{\bf endif};\\
Output $C[T\langle v \rangle]$ as $C(T\langle v \rangle)$. 

\end{tabbing}


\end{document}

\newpage
\section{Counting Paths}
\noindent
{\sc CountPaths}$(G=(N,A))$
\begin{tabbing}
%
% First, we set a tab stop every 3 mm
\hspace{3mm} \= \hspace{3mm} \= \hspace{3mm} \= \hspace{3mm} \= %
\hspace{3mm} \= \hspace{3mm} \= \hspace{3mm} \= \hspace{3mm} \= %
\hspace{3mm} \= \hspace{3mm} \= \hspace{3mm} \= \hspace{3mm}  \kill
{\bf Input:}
A DAG $G = (N,A)$. \\
{\bf Output:} 
For each source $\vv \in N$, the number of paths from $\vv$ to sinks in $G$.\\
Let $N_h$ denote the set of vertices $\vv \in N$ such that 
${\rm ht}(\vv) = h, h \in [0, {\rm ht}(G)]$
\\
$p(\vv) := 1$ for each $\vv \in N_0$; \\
$p(\vv) := 0$ for each $\vv \in N \backslash N_0$;\\
{\bf for each} $h \in [0, \mathrm{ht}(G) -1]$ {\bf do} \+ \\
	{\bf for each} $\vv \in N_h$ {\bf do} \+ \\
		Let $\vv_1,\vv_2,\dots,\vv_k \in N_{h+1}$ such 
			that $\vv_i\vv \in A$ with $\vv_i = \vv + \x_i + \gamma$; \\
		{\bf for each} $i \in [1,k]$ {\bf do}\+ \\
			$p(\vv_i) := x(i)p(\vv_i) + p(\vv)$, \\
			~~~~where $x(i)$ is the number of paths in the DAG corresponding to $\x_i$\- \\
		{\bf end for}\- \\
	{\bf end for} \- \\
{\bf end for}; \\
Output $p(\vv)$, for each $\vv \in N_{{\rm ht(G)}}$,  as
the number of paths from source $\vv$ to sinks in $G$.
\end{tabbing}
- The number of all paths from sources to sinks in $G = (N, A)$
is $\sum_{\vv \in N_{{\rm ht(G)}}} p(\vv)$, where
$p(\vv)$ is the number of paths obtained by 
{\sc CountPaths}$(G=(N,A))$. \\
- Let $G = (N, A)$ and $G' = (N', A')$ denote the two DAGs for core part for a base vertex $e$ and $V_{{\rm pair}}(e)$ denote the set of feasible vector pairs. 
Then the number of all paths that correspond to  
$e$-component is 
$\sum \limits_{\substack{ (\z, \z') \in V_{{\rm pair}}(e)\\
\z \in N, \z' \in N'}} p(\z)p(\z')$ where $p(\z)$ and 
$p(\z')$ are the number of paths from $\z$ and $\z'$ to sinks in 
$G$ and $G'$, respectively.

%%%
%%%%%%%%%%%%%%%%%%%%%%%%%%%%%%%%%%%%%





% \newpage
% % % % % % % % % % % % % % % % % % % % % % % % % % % % % % % % % % % % % % % % % % 
% 
%     Computing Feasible Vector Pairs
% 
% % % % % % % % % % % % % % % % % % % % % % % % % % % % % % % % % % % % % % % % % % 
\subsection{Computing Feasible Vector Pairs for a 
v-component}
For given $ \Delta, \ta, d, m, h$ and frequency vector $\x^*$
of a v-component, a feasible pair  $(\z_1,\z_2)$ is defined to be vectors $\z_1 \in \W_{\co+(\Delta + 1)}^{(0)}(\ta, d-1, m_{\ta},\ta', d_{\ta'}, m_{\ta'}, \delta_2; \x^*)$ and 
$\z_2 \in \W_{\en}^{(\delta_1)}(\tb, d', m'; \x^*)$,\\
where $\delta_1=\lfloor\frac{h - 2-1}{2} \rfloor$
 and $\delta_2 =\lceil \frac{h- 2-1}{2} \rceil$, $ m_{\ta} < m$ such that there exists at least one 
 $\gamma=(\tb\{d'+1\}, \ta'\{d_{\ta'}+1\}, m'') \in \Gamma^{\inn}$
 with 
$m'' \in [1, \min\{3, \val(\ta')-m_{\ta'}, \val(\tb)-m'\}]$ for which it holds that $\x^* = \z_1 + \z_2 + \1_{\gamma}$.
We give a procedure to compute feasible vector pairs for a v-component to generate frequency vectors of rooted core-subtrees  $\W_{\co + \Delta}^{(0)}(\ta, d, m, h; \x^*)$. 


\begin{tabbing}
{\bf Algorithm}~{\sc CombineVertexComp}$(\mbox{global data:~}\ta, d, m, h, \x^*)$\\
{\bf Input:} 
A tuple $(\ta, d, m, h, \x^*)$,
two sets $\W_1$ and $\W_2$ such that for 
$i = 1, 2$, \\ 
\hspace{3mm}
$\W_1 = \W_{\co+(\Delta + 1)}^{(0)}(\ta, d-1, m_{\ta},\ta', d_{\ta'}, m_{\ta'}, \delta_2; \x^*)$ and 
$\W_2 = \W_{\en}^{(\delta_1)}(\tb, d', m'; \x^*)$,\\
\hspace{3mm} where $\delta_1=\lfloor\frac{h - 2-1}{2} \rfloor$
 and $\delta_2 =\lceil \frac{h- 2-1}{2} \rceil$, $ m_{\ta} < m$. \\
{\bf Output:} 
All  feasible pairs $(\z_1,\z_2)$ of vectors
with $\z_i \in \W_i, i=1,2$
\\
\hspace{3mm} and a lower number $q$ on the total number of graph that satisfy all \\
\hspace{3mm} feasible pairs of vectors. \\
%
$q := 0$; \\
{\bf for each} pair   of $\gamma=(\tb\{d'+1\}, \ta'\{d_{\ta'}+1\}, m'') \in \Gamma^{\inn}$
 with \\
 ~~~~~$m'' \in [1, \min\{3, \val(\ta')-m_{\ta'}, \val(\tb)-m'\}]$   {\bf do} \\
 \hspace{3mm} Let
   $L_1$ denote the sorted  list of vectors in $\W_1$; \\
 \hspace{3mm} 
 Construct the set $\overline{\W}:=
 \{\overline{\z} \mid \z \in \W_2\}$
 of  the $\gamma$-complement vectors; \\ 
 \hspace{3mm} Sort the vectors in  $\overline{\W}$ to
obtain a sorted list $L_2$;\\
\hspace{3mm} Merge $L_1$ and $L_2$ into a single sorted list $L_{\gamma}$
of vectors in both lists (as a multiset);\\
\hspace{3mm} Trace the list  $L_{\gamma}$ and for each consecutive pair 
$\z^1, \z^2$ of vectors 
       with $\z^1 = \z^2$\\
    \hspace{8mm} Output $(\z^1, \overline{\z^2})$ as a feasible pair; \\
      \hspace{8mm} Let $T$ be a tree obtained by joining the roots of $T_{\z^1}$ and $T_{\overline{\z^2}}$ with edge- configuration $\gamma$;\\
        \hspace{8mm} $q := q + n_{\z^1} \cdot
      n_{\overline{\z^2}} $\\  
%      	\hspace{8mm} $\delta := \dia^*-4-\delta_1 -1-1$;\\
%      	\hspace{8mm} {\bf if} $\delta =\delta_3$ {\bf then} \\
%      	  \hspace{11mm} {\bf if} $\delta_1= \delta$ 	  {\bf then} \\
%      	   \hspace{14mm}$\ell := \ell + \lceil (n_{\z^1} \cdot n_{\overline{\z^2}} )/6 \rceil$\\      	  
%      	   \hspace{11mm} {\bf else}\\
%      	     \hspace{14mm} $\ell := \ell + \lceil (n_{\z^1} \cdot n_{\overline{\z^2}} )/2 \rceil$\\
%      	      \hspace{11mm} {\bf endif}\\
%      	       \hspace{8mm} {\bf else if} $\delta_1-1= \delta$ 
%      	       {\bf then}\\
%      	        \hspace{11mm} $\ell := \ell + \lceil (n_{\z^1} \cdot n_{\overline{\z^2}} )/2 \rceil$\\
%      	       \hspace{8mm} {\bf else}\\
%        \hspace{11mm} $\ell := \ell + n_{\z^1} \cdot
%       n_{\overline{\z^2}}  $\\
%      \hspace{8mm} {\bf endif}\\      
{\bf endfor}; \\
Output all feasible pairs and $q$ as a lower bound $q$ .

\end{tabbing}

%%%%%%%%%%%%%%
\subsection{Computing Feasible Vector Pairs for an 
e-component}
We give a procedure to compute feasible vector pairs.


\begin{tabbing}
{\bf Algorithm}~{\sc CombineEdgeComp}$(\mbox{global data:~} \x^*, \ell)$\\
{\bf Input:} 
An integer $\ell \geq 2$, 
two sets $\W_1$ and $\W_2$ such that for 
$i = 1, 2$, \\ 
\hspace{3mm}
$\W_i(\ta_i,d_i, m_i, \ta_i^e, 1,m_i^e, h_i; \x^* ) = 
\W_{\co+1, \Delta_i}^{(\delta_i)}(\ta_i,d_i, m_i, \ta_i^e, 1,m_i^e, h_i; \x^* )$,\\
\hspace{3mm} where $\delta_1=\lfloor\frac{\ell - 1}{2} \rfloor$
 and $\delta_2 =\lceil \frac{\ell- 1}{2} \rceil$. \\
{\bf Output:} 
All  feasible pairs $(\z_1,\z_2)$ of vectors
with $\z_i \in \W_i(\ta_i,d_i, m_i), i=1,2$
\\
\hspace{3mm} and a lower number $q$ on the total number of graph that satisfy all \\
\hspace{3mm} feasible pairs of vectors. \\
%
$q := 0$; \\
{\bf for} each pair   of $\gamma=(\ta_1\{d_1+1\}, \ta_2\{d_2+1\}, m) \in \Gamma^{\co}$
 with \\
 ~~~~~$m \in [1, \min\{3, \val(\ta_1)-m_1, \val(\ta_2)-m_2\}]$   {\bf do} \\
 \hspace{3mm} Let
   $L_1$ denote the sorted  list of vectors in $\W_1(\ta_1,d_1, m_1)$; \\
 \hspace{3mm} 
 Construct the set $\overline{\W}:=
 \{\overline{\z} \mid \z \in \W_2(\ta_2, d_2, m_2)\}$
 of  the $\gamma$-complement vectors; \\ 
 \hspace{3mm} Sort the vectors in  $\overline{\W}$ to
obtain a sorted list $L_2$;\\
\hspace{3mm} Merge $L_1$ and $L_2$ into a single sorted list $L_{\gamma}$
of vectors in both lists (as a multiset);\\
\hspace{3mm} Trace the list  $L_{\gamma}$ and for each consecutive pair 
$\z^1, \z^2$ of vectors 
       with $\z^1 = \z^2$\\
    \hspace{8mm} Output $(\z^1, \overline{\z^2})$ as a feasible pair; \\
      \hspace{8mm} Let $T$ be a tree obtained by joining the roots of $T_{\z^1}$ and $T_{\overline{\z^2}}$ with edge- configuration $\gamma$;\\
        \hspace{8mm} $q := q + \lceil (n_{\z^1} \cdot
      n_{\overline{\z^2}} )/2 \rceil$\\  
%      	\hspace{8mm} $\delta := \dia^*-4-\delta_1 -1-1$;\\
%      	\hspace{8mm} {\bf if} $\delta =\delta_3$ {\bf then} \\
%      	  \hspace{11mm} {\bf if} $\delta_1= \delta$ 	  {\bf then} \\
%      	   \hspace{14mm}$\ell := \ell + \lceil (n_{\z^1} \cdot n_{\overline{\z^2}} )/6 \rceil$\\      	  
%      	   \hspace{11mm} {\bf else}\\
%      	     \hspace{14mm} $\ell := \ell + \lceil (n_{\z^1} \cdot n_{\overline{\z^2}} )/2 \rceil$\\
%      	      \hspace{11mm} {\bf endif}\\
%      	       \hspace{8mm} {\bf else if} $\delta_1-1= \delta$ 
%      	       {\bf then}\\
%      	        \hspace{11mm} $\ell := \ell + \lceil (n_{\z^1} \cdot n_{\overline{\z^2}} )/2 \rceil$\\
%      	       \hspace{8mm} {\bf else}\\
%        \hspace{11mm} $\ell := \ell + n_{\z^1} \cdot
%       n_{\overline{\z^2}}  $\\
%      \hspace{8mm} {\bf endif}\\      
{\bf endfor}; \\
Output all feasible pairs and $q$ as a lower bound $q$ .

\end{tabbing}

 
% % % % % % % % % % % % % % % % % % % % % % % % % % % % % % % % % % %  
% 
%             Complete Algorithm for three leaf 2-branches
% 
% % % % % % % % % % % % % % % % % % % % % % % % % % % % % % % % % % % 
\subsection{A Complete Algorithm to compute frequency vectors of v-components}

We briefly summarize how to use the procedures described thus far 
to obtain an algorithm.
Our global constants are a frequency vector $\x^*_{\rm v}$ of a v-component rooted at a base vertex $v$, 
a fixed tuple $(\ta, d, m)$,  
a lower bound 
${\rm ch}_{{\rm LB}}(v)$ and an upper bound
${\rm ch}_{{\rm UB}}(v)$ on core height, where we take 
$\rho = 2$. \bigskip\\
%
\noindent {\sc CompleteAlgorithmVertex}(Global constants: 
$\ta_v, d_v, m_v, \x^*_v$, ${\rm core~height~bounds}$)
\begin{tabbing}
%
% First, we set a tab stop every 3 mm
\hspace{3mm} \= \hspace{3mm} \= \hspace{3mm} \= \hspace{3mm} \= %
\hspace{3mm} \= \hspace{3mm} \= \hspace{3mm} \= \hspace{3mm} \= %
\hspace{3mm} \= \hspace{3mm} \= \hspace{3mm} \= \hspace{3mm}  \kill
% Throughout the description we use \+ and \- to advance and reduce by one tab stop
% the indent of the following lines, and by necessity, use \> to advance the indent 
% of the current line
%
Let $h:= |\Gamma^{\inn}| + 2$;\\
Let $\delta_1 := \lfloor (h - 2-1)/2 \rfloor$,  
$\delta_2 := \lceil (h - 2-1)/2 \rceil$ ;\\
%{\bf for each} base-vertex $v \in V_B$ {\do} \+ \\
	Compute $\W^{(0)}_{\co + \Delta_v}
	(\ta_v, d_v, m_v, h; \x^*_v)$ for a fixed 
	$(\ta_v, d_v, m_v, \Delta_v)$, \\
	~~and for each 
	$h \in [{\rm ch}_{{\rm LB}}(v), 
	\min \{2, {\rm ch}_{{\rm UB}}(v)  \} ]$ if 
	${\rm ch}_{{\rm LB}}(v) \leq 2$ and $\x^*_v(\tt bc)$ = 0;\\
	%
	Compute $\W^{(0)}_{\co + \Delta_v + 1}
	(\ta_v, d_v, m, h; \x^*_v)$ for a fixed 
	$(\ta_v, d_v, \Delta_v)$, \\
	~~for each $m \in [d_v -1, \val(\ta_v) - \Delta_v -1]$, 
	$h \leq 2$ if 
	${\rm ch}_{{\rm UB}}(v) > 2$ and $\x^*_v(\tt bc)$ = 1;\\
	%
	Compute 
	$\W_{\en}^{(0)}(\ta, d, m; \x^*_v)$ for each 
$\ta \in \Lambda$, $d \in [1, \val(\ta)-1]$, \\
~~$m \in [d, \val(\ta)-1]$ if 
	${\rm ch}_{{\rm UB}}(v) > 2$ and $\x^*_v(\tt bc)$ = 1;\\
	%
	Compute $\W_{\inl}^{(0)}(\ta, d, m; \x^*_v)$ for each 
	$\ta \in \Lambda$, $d \in [0, \val(\ta)-2]$, \\
~~$m \in [d, \val(\ta)-2]$
	if 
	${\rm ch}_{{\rm UB}}(v) > 2$ and $\x^*_v(\tt bc)$ = 1;\\
	%
	Compute 
	$\W_{\en}^{(\delta_1)}(\tb, d', m'; \x^*_v)$ for each 
$\tb \in \Lambda$, $d' \in [1, \val(\tb)-1]$, \\
~~$m' \in [d', \val(\tb)-1]$, 
     if 
	${\rm ch}_{{\rm UB}}(v) > 2$ and $\x^*_v(\tt bc)$ = 1;\\
	%	
	Compute $\W_{\co+(\Delta + 1)}^{(0)}(\ta_v, d_v-1, m_{\ta}, \ta', d_{\ta'}, m_{\ta'}, \delta_2; \x_v^*)$ , for 
$\Delta \in [2, 3]$,  
$\ta, \ta' \in \Lambda$, \\
~~integers
$d_{\ta} \in [0, \val(\ta) - \Delta-1]$, 
$m_{\ta} \in [d_{\ta}, \val(\ta) - \Delta-1]$, $m_{\ta_v} < m_{v}$,
$d_{\ta'} \in [1, \val(\ta') - 1]$, \\
~~$m_{\ta'} \in [d_{\ta'}, \val(\ta')-1]$, if ${\rm ch}_{{\rm UB}}(v) > 2$ and $\x^*_v(\tt bc)$ = 1;\\
{\bf for each} two tuples $(\ta_v, d_v-1, m_{\ta}, \ta', d_{\ta'}, m_{\ta'}, \delta_2; \x_v^*)$, $(\tb, d', m'; \x^*_v)$ {\bf do} \+ \\
      search for a feasible vector pair
            in the pair of sets\\ 
            $\W_{\co+(\Delta + 1)}^{(0)}(\ta_v, d_v-1, m_{\ta}, \ta', d_{\ta'}, m_{\ta'}, \delta_2; \x_v^*)$ and  $\W_{\en}^{(\delta_1)}(\tb, d', m'; \x^*_v)$ \- \\
    {\bf end for}.
%	Compute 
%	$\W_{\co + \Delta}^{(0)}(\ta, d, m, h; \x^*_v)$ for each 
%     $\Delta \in [2, 3]$, 
%     $\ta \in \Lambda$, $d \in [1, \val(\ta)-1]$, \\
%~~$m \in [d, \val(\ta)-1]$, 
%     $h = \text{number of internal edges in $\x^*_v$} + 2$, 
%     if 
%	${\rm ch}_{{\rm UB}}(v) > 2$ and $\x^*_v(\tt bc)$ = 1;\\
	%	
%{\bf end for}
\end{tabbing}
%%%%%%%%%%%%%%%%%%
% \newpage
% % % % % % % % % % % % % % % % % % % % % % % % % % % % % % % % % % % % % % % % % 
% 
%                 Getting Vectors of bi-rooted core-subtrees
% 
% % % % % % % % % % % % % % % % % % % % % % % % % % % % % % % % % % % % % % % % % 
\subsection{Generation of Frequency Vectors of Bi-rooted Core-subtrees}
For an integer $h \in [h_1, h_2]$, elements
$\ta, \ta^{e} \in \Lambda$, integers
$d  \in [1, \val(\ta) - 1]$, 
$m\in [d,\val(\ta)-1]$, 
$\Delta^e \in [1, \val(\ta^e) -1] $, 
$m^e \leq \val(\ta^e) - \Delta^e$,
and $q \geq 1$, 
 we give a procedure to 
compute the set $\W_{\co+1, \Delta^e}^{(q)}(\ta, d, m, \ta^e, 1, m^e, h; \x^*)$.
\bigskip\\
\noindent {\sc ComputeBiRootedCoreSubtree($\ta, d, m, \ta^e, 1, m^e, h, q$)}
\begin{tabbing}
%
% First, we set a tab stop every 3 mm
\hspace{3mm} \= \hspace{3mm} \= \hspace{3mm} \= \hspace{3mm} \= %
\hspace{3mm} \= \hspace{3mm} \= \hspace{3mm} \= \hspace{3mm} \= %
\hspace{3mm} \= \hspace{3mm} \= \hspace{3mm} \= \hspace{3mm}  \kill
% Throughout the description we use \+ and \- to advance and reduce by one tab stop
% the indent of the following lines, and by necessity, use \> to advance the indent 
% of the current line
%

%%% Important %% In the next 2 can be replaced by \rho to 
%%% make the algorithm general
{\bf Input:}
An integer $h\geq 0$, elements  
$\ta, \ta^{e} \in \Lambda$, integers
$d  \in [1, \val(\ta) - 1]$, 
$m\in [d,\val(\ta)-1]$,  \+ \\
$\Delta^e \in [1, \val(\ta^e) -1] $, 
$m^e \leq \val(\ta^e) - \Delta^e$,
and $q \geq 1$. \\
\hspace{3mm}  /* Global data: A vector $\x^* = (\x^*_{\co}, \x^*_{\inn}, \x^*_{\ex}, b)$ with 
$\x^*_{\co} \in \mathbb{Z}^{\Lambda^{\co}}$, 
$\x^*_{{\tt t}} \in \mathbb{Z}^{\Lambda^{{\tt t}}}$,
${\tt t} \in \{\inn, \ex\}$, \+ \\
a non-negative integer $b$,
                 the collection \\
                 $\Ww_{\co+ 2}^{(0)}$  vector sets 
                 $\W_{\co + 2}^{(0)}(\ta, d-1, m_{\ta}, p; 
                 \x^*)$, 
                 %$d_{\ta} \in [0, \val(\ta)-3]$,
                 $m_{\ta} \in [d-1, \val(\ta) -\Delta -1]$, 
                 $p \in [0,h]$,\\
                 for $q \geq 2$, 
                 $\Ww_{\en}^{(q-1)}$ of vector sets  
                 $\W_{\co+1, \Delta^e}^{(q-1)}
                 (\tb, d', m', \ta^e, 1, m^e, h'; \x^*)$, \\
                 $\tb \in \Lambda$,
                  $d' \in [1, \val(\tb)-1]$, 
                  $m' \in [d', \val(\tb) - 1]$, 
                  $h' \in [0, h]$, integer $g \geq 1$ 
                  and \\
                  for each vector $w$ in these sets, 
                  a set $\mathcal{T}_{\w}$ of sample trees  
      $T_{\w}$ of size at most $g$ and \\and the number $n_{\w}$  of 
                  samples trees\\
                   with vector $\w$. */ \- \\
{\bf Output:} The set $\W_{\co+1, \Delta^e}^{(q)}(\ta, d, m, \ta^e, 1, m^e, h; \x^*)$, where 
we store each vector \+ \\
$\w \in \W_{\co+1, \Delta^e}^{(q)}(\ta, d, m, \ta^e, 1, m^e, h; \x^*)$,\\
       a set $\mathcal{T}_{\w}$ of sample trees  
      $T_{\w}$ of size at most $g$ and \\
number $n_{\w}$ of trees with vector $\w$ in a trie.\- \\
$\W := \emptyset$; \\
{\bf for each} triplet $(\ta, d-1, m_{\ta}, p)$ {\bf do} \+ \\
	{\bf if} $q = 1$ {\bf then}\+ \\
		{\bf if} $p = h$ and 
	     $\val(\ta) \geq m_{\ta} + m^e$ {\bf then} \+ \\
	     	{\bf for each} $\w^{\ta}  \in 
                      \W_{\inl}^{(0)}(\ta, d-1,m_{\ta}, p; \x^*)$ 
                      {\bf do} \+\\                
				$\gamma^{\co} := (\ta d, \ta^e 1, m^e )$; 
				$\w := \w^{\ta} + \1_{\gamma^{\co}}$\\
				  {\bf if}  $\gamma^{\co} \in \Gamma^{\co}$ and 
						$\w \leq \x^*$ {\bf then}\+ \\
						                     {\bf if} $\w \in 
                     \W$ {\bf then} 
                $n_{\w}  = n_{\w} +  n_{\w^{\ta}} \cdot n_{\w^{\tb}}$\\
               {\bf else}      \+ \\                         
                     $\W := \W \cup \{ \w \}$; 
                     $\mathcal{T}_{\w}: =\emptyset$; 
                     $n_{\w} := n_{\w^{\ta}} \cdot n_{\w^{\tb}}$ \- \\
                       {\bf end if}  \\
                        {\bf if} $\w \in 
                     \W$ {\bf then} \+ \\
                     {\bf for each } $T_{\w^{\ta}} \in 
                     \mathcal{T}_{\w^{\ta}}$ and $T_{\w ^{\tb}}\in 
                     \mathcal{T}_{\w^{\tb}}$ {\bf do} \+ \\
                     	Let $T$ be the tree obtained by 
                      joining the roots of 
			  $T_{\w^{\ta}}$ and $T_{\w ^{\tb}}$ \\
			 ~~ by an edge of multiplicity $m'$;\\
			 {\bf if } $|\mathcal{T}_{\w}| < g$ {\bf then}
 				$\mathcal{T}_{\w}: = \mathcal{T}_{\w} \cup 
 				\{T\}$\- \\
                     {\bf end for} \- \\
                      {\bf end if} \- \\
				{\bf end if} \- \\
			{\bf end for} \- \\
		{\bf end if} \- \\
	{\bf else} /* $q>1$ */ \+ \\
		{\bf for each} triplet  $(\tb, d_{\tb}, m_{\tb}, h')$ {\bf do} \+ \\
		  {\bf for each} $\w^{\tb} \in 
   \W_{\co+1, \Delta^e}^{(q-1)}
                 (\tb,d_{\tb}, m_{\tb}, \ta^e, 1, m^e, h'; \x^*)$
                         {\bf do} \+ \\
            {\bf for each} $m' \in [1, 3]$ such that \+  \\ 
                      - $\gamma^{\co} := (\ta d, \tb  \{d_{\tb} + 1\},
                       m') \in \Gamma^{\co}$ and\\
                      - $m_{\ta} + m' = m, 
                      m_{\ta} + m' + 1 \leq \val(\ta)$, 
                      $m' +m_{\tb} \leq \val(\tb)$, \\
                      - $h = \max\{p, h'\}$ and \\
                      - $\w:= \w_{\ta} + \w_{\tb} + 
                      \1_{\gamma^{\co}} \leq \x^*$
                                             {\bf do} \+ \\
                                      {\bf if} $\w \in 
                     \W$ {\bf then} 
                $n_{\w}  = n_{\w} +  n_{\w^{\ta}} \cdot n_{\w^{\tb}}$\\
               {\bf else}      \+ \\                         
                     $\W := \W \cup \{ \w \}$; 
                     $\mathcal{T}_{\w}: =\emptyset$; 
                     $n_{\w} := n_{\w^{\ta}} \cdot n_{\w^{\tb}}$ \- \\
                       {\bf end if}  \\
                        {\bf if} $\w \in 
                     \W$ {\bf then} \+ \\
                     {\bf for each } $T_{\w^{\ta}} \in 
                     \mathcal{T}_{\w^{\ta}}$ and $T_{\w ^{\tb}}\in 
                     \mathcal{T}_{\w^{\tb}}$ {\bf do} \+ \\
                     	Let $T$ be the tree obtained by 
                      joining the roots of 
			  $T_{\w^{\ta}}$ and $T_{\w ^{\tb}}$ \\
			 ~~ by an edge of multiplicity $m'$;\\
			 {\bf if } $|\mathcal{T}_{\w}| < g$ {\bf then}
 				$\mathcal{T}_{\w}: = \mathcal{T}_{\w} \cup 
 				\{T\}$\- \\
                     {\bf end for} \- \\
                      {\bf end if} \- \\  
		{\bf end for} \- \\  
		{\bf end for} \- \\
		{\bf end for} \- \\
	{\bf end if} \- \\
{\bf end for;} \\
Output $\W$ as 
$\W_{\co+1, \Delta^e}^{(q)}(\ta, d, m, \ta^e, 1, m^e, h; \x^*)$,
and for each $\w \in \W$, $\mathcal{T}_{\w}$ and $n_{\w}$.
\end{tabbing}



\newpage
\subsection{A Complete Algorithm to compute frequency vectors of e-components}

We briefly summarize how to use the procedures described thus far 
to obtain an algorithm.
Our global constants are a frequency vector $\x^*_{\rm e}$ of an e-component, 
two fixed tuples $(\ta^e_j, m^e_j, \Delta^e_j), j = 1, 2$
a lower bound 
${\rm ch}_{{\rm LB}}(e)$ and an upper bound
${\rm ch}_{{\rm UB}}(e)$ on core height, where we take 
$\rho = 2$. \bigskip\\
%
\noindent {\sc CompleteAlgorithmEdge}(Global constants: 
$\ta^e_j, m^e_j, \Delta^e_j, \x^*_e, {\rm core~height~bounds}$)
\begin{tabbing}
%
% First, we set a tab stop every 3 mm
\hspace{3mm} \= \hspace{3mm} \= \hspace{3mm} \= \hspace{3mm} \= %
\hspace{3mm} \= \hspace{3mm} \= \hspace{3mm} \= \hspace{3mm} \= %
\hspace{3mm} \= \hspace{3mm} \= \hspace{3mm} \= \hspace{3mm}  \kill
% Throughout the description we use \+ and \- to advance and reduce by one tab stop
% the indent of the following lines, and by necessity, use \> to advance the indent 
% of the current line
%{\bf for each} base-edge $e \in E_B$ {\do} \+ \\
$\Gamma^{\inn}_e: = \text{The set internal edges in 
	$\x^*_e$}$;\\
Compute $\W^{(0)}_{\co + \Delta}
	(\ta, d, m, h; \x_e^*)$ for each \\
		~~
		$\Delta \in [2,3]$, 
		$\ta \in \Lambda$, $d \in [0, \val(\ta)-\Delta]$,
		$m \in [d, \val(\ta) -\Delta]$, 
	$h \in [0, 	\min \{2, {\rm ch}_{{\rm UB}}(e)  \} ]$;\\
	%
	Compute 
	$\W_{\en}^{(0)}(\ta, d, m; \x^*_e)$ for each 
	$\ta \in \Lambda$, $d \in [1, \val(\ta)-1]$,
 	$m \in [d, \val(\ta)-1]$;\\
	%
	Compute $\W_{\inl}^{(0)}(\ta, d, m; \x^*_e)$ for each 
	$\ta \in \Lambda$, $d \in [0, \val(\ta)-2]$, 
	$m \in [d, \val(\ta)-2]$;\\
	%
	
	Compute 
	$\W_{\en}^{(h)}(\ta, d, m; \x^*_e)$ for each 
	$\ta \in \Lambda$, $d \in [1, \val(\ta)-1]$, \\
~~$m \in [d, \val(\ta)-1]$, 
     $h = \min\{|\Gamma^{\inn}_e| -1, 
     {\rm ch}_{{\rm UB}}(e)-2-1\}$
     if 
	${\rm ch}_{{\rm UB}}(e) > 2$;\\
	%
	Compute 
	$\W_{\co + \Delta}^{(0)}(\ta, d, m, h; \x^*_v)$ for each 
     $\Delta \in [2, 3]$, 
     $\ta \in \Lambda$, $d \in [1, \val(\ta)-1]$, \\
~~$m \in [d, \val(\ta)-1]$, 
     $h = \min\{|\Gamma^{\inn}_e| +2, 
     {\rm ch}_{{\rm UB}}(e)\}$, 
     if 
	${\rm ch}_{{\rm UB}}(e) > 2$ ;\\
	%
	Compute $\W_{\co+1, \Delta^e_j}^{(q)}(\ta, d, m, \ta^e_j, 1,
	 m^e_j, h; \x^*_e)$ for fixed 
	$(\ta^e_j, m^e_j, \Delta^e_j)$, 
	$\ta, \in \Lambda$, \\
	~~integers
	$d  \in [1, \val(\ta) - 1]$, 
	$m\in [d,\val(\ta)-1]$, 
	$q = \Delta_j ^e, j = 1, 2$;\\
	%
	{\bf for each} two tuples $(\ta_j, d_j, m_j, \ta^e_j, 1,
	 m^e_j, h; \x^*_e)$, $j = 1, 2$ {\bf do} \+ \\
      search for a feasible vector pair
            in the pair of sets 
            $\W_{\co+1, \Delta^e_j}^{(q)}(\ta_j, d_j, m_j, 
            \ta^e_j, 1,  m^e_j, h_j; \x^*_e)$  \- \\
            
{\bf end for}.\\
\end{tabbing}

%%%%%%%%%%%%%%%%%%%%
\pagebreak 
\subsection{A Complete Algorithm to compute frequency vectors of isomers of a given graph $G^{\dagger}$}

We briefly summarize how to use the procedures described thus far 
to obtain an algorithm.
Our global constants are a  graph $G^{\dagger}$, 
a path partition $\mathcal{P} =\{P_1, P_2, \ldots, P_p\}$, and 
for each base-vertex $t$ or base-edge $t$, a lower bound 
${\rm ch}_{{\rm LB}}(t)$ and an upper bound
${\rm ch}_{{\rm UB}}(t)$ on core height,where we take 
~$\rho = 2$. \\
%
\noindent {\sc CompleteAlgorithm}(Global constants: $G^{\dagger},\mathcal{P}, {\rm core~height~bounds}$)
\begin{tabbing}
%
% First, we set a tab stop every 3 mm
\hspace{3mm} \= \hspace{3mm} \= \hspace{3mm} \= \hspace{3mm} \= %
\hspace{3mm} \= \hspace{3mm} \= \hspace{3mm} \= \hspace{3mm} \= %
\hspace{3mm} \= \hspace{3mm} \= \hspace{3mm} \= \hspace{3mm}  \kill
% Throughout the description we use \+ and \- to advance and reduce by one tab stop
% the indent of the following lines, and by necessity, use \> to advance the indent 
% of the current line
%
Let $\ell_i := \ell(P_i)$ for each $i \in [1, p]$;\\
Let $\delta_1^i := \lfloor (\ell_i - 1)/2 \rfloor$,  
$\delta_2^i := \lceil (\ell_i - 1)/2 \rceil$ for each $i \in [1, p]$ ;\\
Compute frequency vector $\x^*_t$ for each base-vertex 
$t$ or and base-edge $t$;\\
{\bf for each} base-vertex $v \in V_B$ {\do} \+ \\
Let $h:= |\Gamma^{\inn}| + 2$;\\
Let $\delta'_1 := \lfloor (h - 2-1)/2 \rfloor$,  
$\delta'_2 := \lceil (h - 2-1)/2 \rceil$ ;\\
	Compute $\W^{(0)}_{\co + \Delta_v}
	(\ta_v, d_v, m_v, h; \x^*_v)$ for a fixed 
	$(\ta_v, d_v, m_v, \Delta_v)$, \\
	~~and for each 
	$h \in [{\rm ch}_{{\rm LB}}(v), 
	\min \{2, {\rm ch}_{{\rm UB}}(v)  \} ]$ if 
	${\rm ch}_{{\rm LB}}(v) \leq 2$ and $\x^*_v(\tt bc)$ = 0;\\
	%
	Compute $\W^{(0)}_{\co + \Delta_v + 1}
	(\ta_v, d_v, m, h; \x^*_v)$ for a fixed 
	$(\ta_v, d_v, \Delta_v)$, \\
	~~for each $m \in [d_v -1, \val(\ta_v) - \Delta_v -1]$, 
	$h \leq 2$ if 
	${\rm ch}_{{\rm UB}}(v) > 2$ and $\x^*_v(\tt bc)$ = 1;\\
	%
	Compute 
	$\W_{\en}^{(0)}(\ta, d, m; \x^*_v)$ for each 
$\ta \in \Lambda$, $d \in [1, \val(\ta)-1]$, \\
~~$m \in [d, \val(\ta)-1]$ if 
	${\rm ch}_{{\rm UB}}(v) > 2$ and $\x^*_v(\tt bc)$ = 1;\\
	%
	Compute 
	$\W_{\en}^{(\delta'_1)}(\tb, d', m'; \x^*_v)$ for each 
$\tb \in \Lambda$, $d' \in [1, \val(\tb)-1]$, \\
~~$m' \in [d', \val(\tb)-1]$, 
     if 
	${\rm ch}_{{\rm UB}}(v) > 2$ and $\x^*_v(\tt bc)$ = 1;\\
	%	
	Compute $\W_{\co+(\Delta + 1)}^{(0)}(\ta_v, d_v-1, m_{\ta}, \ta', d_{\ta'}, m_{\ta'}, \delta'_2; \x_v^*)$ , for 
$\Delta \in [2, 3]$,  
$\ta, \ta' \in \Lambda$, \\
~~integers
$d_{\ta} \in [0, \val(\ta) - \Delta-1]$, 
$m_{\ta} \in [d_{\ta}, \val(\ta) - \Delta-1]$, $m_{\ta_v} < m_{v}$,
$d_{\ta'} \in [1, \val(\ta') - 1]$, \\
~~$m_{\ta'} \in [d_{\ta'}, \val(\ta')-1]$, if ${\rm ch}_{{\rm UB}}(v) > 2$ and $\x^*_v(\tt bc)$ = 1;\\
{\bf for each} two tuples $(\ta_v, d_v-1, m_{\ta}, \ta', d_{\ta'}, m_{\ta'}, \delta'_2; \x_v^*)$, $(\tb, d', m'; \x^*_v)$ {\bf do} \+ \\
      search for a feasible vector pair
            in the pair of sets\\ 
            $\W_{\co+(\Delta + 1)}^{(0)}(\ta_v, d_v-1, m_{\ta}, \ta', d_{\ta'}, m_{\ta'}, \delta'_2; \x_v^*)$ and  $\W_{\en}^{(\delta'_1)}(\tb, d', m'; \x^*_v)$ \- \\
    {\bf end for}\- \\
	%	
{\bf end for};\\
%
{\bf for each} base-edge $e \in E_B$ {\do} \+ \\
Compute $\W^{(0)}_{\co + \Delta}
	(\ta, d, m, h; \x_e^*)$ for each \\
		~~
		$\Delta \in [2,3]$, 
		$\ta \in \Lambda$, $d \in [0, \val(\ta)-\Delta]$,
		$m \in [d, \val(\ta) -\Delta]$, 
	$h \in [0, 	\min \{2, {\rm ch}_{{\rm UB}}(e)  \} ]$;\\
	%
	Compute 
	$\W_{\en}^{(0)}(\ta, d, m; \x^*_e)$ for each 
	$\ta \in \Lambda$, $d \in [1, \val(\ta)-1]$,
 	$m \in [d, \val(\ta)-1]$;\\
	%
	Compute $\W_{\inl}^{(0)}(\ta, d, m; \x^*_e)$ for each 
	$\ta \in \Lambda$, $d \in [0, \val(\ta)-2]$, 
	$m \in [d, \val(\ta)-2]$;\\
	%
	$\Gamma^{\inn}_e: = \text{number of internal edges in 
	$\x^*_e$}$;\\
	Compute 
	$\W_{\en}^{(h)}(\ta, d, m; \x^*_e)$ for each 
	$\ta \in \Lambda$, $d \in [1, \val(\ta)-1]$, \\
~~$m \in [d, \val(\ta)-1]$, 
     $h = \min\{|\Gamma^{\inn}_e| -1, 
     {\rm ch}_{{\rm UB}}(e)-2-1\}$
     if 
	${\rm ch}_{{\rm UB}}(e) > 2$;\\
	%
	Compute 
	$\W_{\co + \Delta}^{(0)}(\ta, d, m, h; \x^*_v)$ for each 
     $\Delta \in [2, 3]$, 
     $\ta \in \Lambda$, $d \in [1, \val(\ta)-1]$, \\
~~$m \in [d, \val(\ta)-1]$, 
     $h = \min\{|\Gamma^{\inn}_e| +2, 
     {\rm ch}_{{\rm UB}}(e)\}$, 
     if 
	${\rm ch}_{{\rm UB}}(e) > 2$ ;\\
	%
	Compute $\W_{\co+1, \Delta^e_j}^{(q)}(\ta, d, m, \ta^e_j, 1,
	 m^e_j, h; \x^*_e)$ for fixed 
	$(\ta^e_j, m^e_j, \Delta^e_j)$, 
	$\ta, \in \Lambda$, \\
	~~integers
	$d  \in [1, \val(\ta) - 1]$, 
	$m\in [d,\val(\ta)-1]$, 
	$q = \Delta_j ^e, j = 1, 2$;\\
	%
	{\bf for each} two tuples $(\ta_j, d_j, m_j, \ta^e_j, 1,
	 m^e_j, h; \x^*_e)$, $j = 1, 2$ {\bf do} \+ \\
      search for a feasible vector pair
            in the pair of sets 
            $\W_{\co+1, \Delta^e_j}^{(q)}(\ta_j, d_j, m_j, 
            \ta^e_j, 1,  m^e_j, h_j; \x^*_e)$  \\
            /* We store feasible pairs with $\max \{h_1, h_2\} = 
   {\rm ch}(G^\dagger)$ and\\
  ~~feasible pairs $\max \{h_1, h_2\} \neq 
    {\rm ch}(G^\dagger)$ in different sets */\- \\
    {\bf end for}\-\\
{\bf end for}.\\
\end{tabbing}
%\end{document}  
 