%%%%%%%%%%%%%%%%%%%
%
% File: Module4_manual_jp.tex
% Japanese manual for the program for generating chemical isomers
% Edited: Feb 8, 2021, Alex
%
%%%%%%%%%%%%%%%%%%%%

\documentclass[11pt,titlepage,dvipdfmx,twoside]{jarticle}
\linespread{1.1}

\usepackage{amsfonts}
\usepackage{amssymb}
\usepackage{amsmath}
\usepackage{amsthm}
\newtheorem{theorem}{Theorem}
\newtheorem{lemma}[theorem]{Lemma}
\usepackage{enumitem}
\usepackage{geometry}
\geometry{left=2.5cm,right=2.5cm,top=2.5cm,bottom=2.5cm}

\usepackage{algorithm}  
\usepackage{algorithmic}  
\renewcommand{\algorithmicrequire}{\textbf{Input:}} 
\renewcommand{\algorithmicensure}{\textbf{Output:}}
\renewcommand{\algorithmicforall}{\textbf{for each}}

\usepackage{mathtools}
\usepackage{comment}
\usepackage[dvipdfmx]{graphicx}
\usepackage{float}
\usepackage{framed}
\usepackage{graphicx}
\usepackage{subcaption}
\usepackage{listings}
\usepackage{color}
\usepackage{url}

\definecolor{codegreen}{rgb}{0,0.6,0}
\definecolor{codegray}{rgb}{0.5,0.5,0.5}
\definecolor{codepurple}{rgb}{0.58,0,0.82}
\definecolor{backcolour}{rgb}{0.95,0.95,0.92}
 
\newcommand{\project}{{\tt mol-infer/Cyclic}}

\title{\huge{Module 4における入力した2-Lean化学グラフの構造異性体を生成するプログラムの使用説明書}}

\author{\project}

\begin{document}

% The following makeatletter must be after begin{document}
\makeatletter 
\let\c@lstlisting\c@figure
\makeatother

\西暦
\date{\today}

\maketitle

% \cleardoublepage

\thispagestyle{empty}
\tableofcontents
\clearpage

\pagenumbering{arabic}

\section{概要}
この冊子では,与えられた2-lean環状化学グラフの構造異性体を列挙する~\cite{branch}プログラムの使い方を説明する.

また,このフォルダにはこの冊子以外に以下のファイルとフォルダが含まれている.
%
	\begin{itemize}
		
	\item フォルダ{\tt main}\\
		プログラムを走らせるmainフォルダである.
		\begin{itemize}
			\item{gen\_partition.cpp}\\
				環状化学グラフを, 非環状部分グラフに分割するためのプログラムである.
				
			\item{main.cpp}\\
				2-lean環状化学グラフの構造異性体を列挙するためのプログラムである.

			\item{Presudocode\_Graph\_Generation.pdf}\\
			グラフ探索アルゴリズムの疑似コードを記したpdfファイルである.
		\end{itemize}
		
	\item フォルダ{\tt include}\\
		このプログラムに使われるヘッダファイルのフォルダである.
		\begin{itemize}
			\item{chemical\_graph.hpp}\\
			  化学グラフためのヘッダファイルである.
			\item{cross\_timer.h}\\
				計算時間を計測するためのヘッダファイルである.
				
			\item{data\_structures.hpp}\\
				化学グラフのデータ構造を定義するヘッダファイルである. 
				
			\item{debug.h}\\
				デバッグのためのヘッダファイルである.
				
			\item{fringe\_tree.hpp}\\
				外縁木(fringe tree)~\cite{branch}を列挙する関数のヘッダファイルである.
				
			\item{tools.hpp}\\
				便利なツール関数のヘッダファイルである.		
		\end{itemize}
	\item フォルダ{\tt instances}\\
		入力のインスタンスのフォルダである.
		\begin{itemize}
			\item{\tt sample\_1.sdf}\\
				節点数が20, 核サイズが18,核高が1
				となる化合物が記載されている入力データである.
			\item{\tt sample\_1\_partition.txt}\\
				sample\_1の分割情報ファイルである.
				\item{\tt sample\_2.sdf}\\
				節点数が50, 核サイズが24,核高が6
				となる化合物が記載されている入力データである.
			\item{\tt sample\_2\_partition.txt}\\
			sample\_2の分割情報ファイルである.
			\item{\tt sample\_3.sdf}\\
			節点数が60, 核サイズが31,核高が4
			となる化合物が記載されている入力データである.
			\item{\tt sample\_3\_partition.txt}\\
			sample\_3の分割情報ファイルである.
			\item{\tt sample\_4.sdf}\\
			節点数が120, 核サイズが60,核高が4
			となる化合物が記載されている入力データである.
			\item{\tt sample\_4\_partition.txt}\\
			sample\_4の分割情報ファイルである.

		\end{itemize}

	\end{itemize}

次に,この冊子の構成について説明する.
第\ref{sec:term}節では,この冊子およびプログラム内で使用している用語について説明する.
第\ref{sec: partition}節では, 環状化学グラフを非環状部分グラフに分割するプログラムの入力と出力について説明し, 実際の計算例を示す.
第\ref{sec: main}節では,与えられた2-lean環状化学グラフの構造異性体を生成するプログラムの入力と出力について説明し, 実際の計算例を示す.


\newpage

\section{用語の説明}
\label{sec:term}

この節では,用語について説明する.
\begin{itemize}

\item 化学グラフ\\
	化学グラフは節点集合と枝集合の組で化合物の構造を表すものである.
	各節点には原子の種類が割り当てられており,各枝には結合の多重度が割り当てられている.
	この冊子では水素原子が省略された化学グラフを扱っていく.


\item 特徴ベクトル\\
化学グラフにおける各化学元素の数などの情報を与える数値ベクトル.
本プロジェクトの特徴ベクトルに使われるディスクリプターの詳細な情報は, ~\cite{branch}を参照.


\item 分割情報\\
入力する化学グラフの基底節点・枝 (base vertex/edge) の指定とその節点・枝成分(vertex/edge component)[1]を固定するか可変にするかの情報.
より詳細な情報は, ~\cite{branch}を参照.


\end{itemize}

\section{非環状部分グラフへ分割するプログラム}
\label{sec: partition}

\subsection{入力と出力}
\label{sec:InOut_p}

この節では, 環状化学グラフを非環状部分グラフに分割するために使うプログラムの入力と出力情報にすいて説明する.
以下ではこのプログラムのことを{\bf 分割プログラム}と呼ぶ.
\ref{sec:Input_p}節では,プログラムの入力情報について説明する.
%\ref{sec:InputFormat_p}節では,計算機上で入力する際のデータ形式について説明する.
\ref{sec:Output_p}節では,プログラムの出力情報について説明する.
\ref{sec:OutputFormat_p}節では,計算機上でプログラムを実行した際に出力されるデータ形式について説明する.

\subsubsection{プログラムの入力}
\label{sec:Input_p}

分割プログラムでは二つの情報を入力とする.
一つ目は化学グラフである.
化学グラフはSDFファイルの形式で入力する.
%また,入力の方法については第\ref{sec:Example}節を参照すること.
%\bigskip
%\newpage
%\subsubsection{入力データの形式}
%\label{sec:InputFormat_p}
この分割プログラムの入力するSDFファイルのフォーマットについて,
\url{https://www.chem-station.com/blog/2012/04/sdf.html}などの解説が分かりやすい.
さらに,正確な定義書として,公式資料 \url{http://help.accelrysonline.com/ulm/onelab/1.0/content/ulm_pdfs/direct/reference/ctfileformats2016.pdf} を参照するとよい.

二つ目は出力される分割情報を保存するtxtファイル名である.
\bigskip

\subsubsection{プログラムの出力}
\label{sec:Output_p}

分割プログラムの出力は,
入力された化学グラフの分割情報である.
分割情報はファイル名が入力されたtxtファイルに出力される.

\subsubsection{出力データの形式}
\label{sec:OutputFormat_p}

\begin{oframed}
	{\bf 出力データの形式}\\\\
	{\tt 4 \\
	7 \# C \\
	0 0 0 \\
	15 \# C \\
	0 0 0 \\
	10 \# C \\
	0 0 0 \\
	16 \# C \\
	0 0 0 \\
	5 \\
	7 4 3 5 6 9 2 15 \# C1C1N1C1C1C1O1C \\
	0 1 0 \\
	7 10 \# C2C \\
	0 0 1 \\
	10 12 14 13 11 7 \# C1C2C1C2C1C \\
	0 0 1 \\
	15 16 \# C2C \\
	0 0 1 \\
	16 18 20 19 17 15 \# C1C2C1C2C1C \\
	0 0 1 \\}
	\end{oframed}

	この具体例を用いて,各行の内容を説明する.
	数値例とそれぞれの内容の対応を表~\ref{tab:PartitionFormat}に示す.

\bigskip
\begin{table}[H]
\begin{center} \caption{出力ファイルの構成}
\label{tab:PartitionFormat}
  \begin{tabular}{l|l}
  数値例 & 内容\\ \hline \hline
{\tt  4} & 基底節点の数\\ \hline
{\tt  7 \# C} & 入力したSDFファイル内での, 基底節点の指標とその元素\\
{\tt  0 0 0} & 核高の下限と上限, \\
{\tt  15 \# C} & \hspace{10mm}そして節点成分を固定するかしないか (0/1)\\
{\tt  0 0 0} & \\
{\tt  10 \# C} & \\
{\tt  0 0 0} & \\ 
{\tt  16 \# C} &\\ 
{\tt  0 0 0} & \\ \hline
{\tt  5} & 基底枝の数 \\ \hline
{\tt  7 4 3 5 6 9 2 15 \# C1C1N1C1C1C1O1C} & 入力したSDFファイル内での, 基底枝の指標とその元素,\\
{\tt  0 1 0} & \hspace{10mm}そして隣り合う元素同士の価数 \\
{\tt  7 10 \# C2C} &\\ 
{\tt  0 0 1} & 核高の下限と上限,\\
{\tt  10 12 14 13 11 7 \# C1C2C1C2C1C} & \hspace{10mm}そして枝成分を固定するかしないか (0/1)\\\\
{\tt  0 0 1} & \\
{\tt  15 16 \# C2C} & \\
{\tt  0 0 1} & \\
{\tt  16 18 20 19 17 15 \# C1C2C1C2C1C} & \\
{\tt  0 0 1} & \\ \hline
  \end{tabular}
\end{center}
\end{table}

%\newpage

%\section{プログラムの説明}
%\label{sec:section4}
%この節ではプログラムの流れについて説明する.

%\newpage

\subsection{プログラムの実行と計算例}
\label{sec:Example_p}

ここでは分割プログラムの実行方法と,具体例を入力した場合の計算結果を示す.

\subsubsection{実行方法}
\label{sec:compile_p}
\begin{itemize}
	\item {\em 環境確認}\\
		ISO C++ 2011標準に対応するC++コンパイラーがあれば問題ないと考えられる.
		%
		Linux Ubuntu 20.04, コンパイラー g++ ver 9.3 で確認したが,g++がインストールされていない場合は,次のようにインストールできる.\\
		\verb|$ sudo apt install g++|
	\item {\em コンパイル}\\
		\verb|$ g++ -o gen_partition gen_partition.cpp -O3 -std=c++11|
	\item {\em 実行}\\
		\verb|$ ./gen_partition instance.sdf instance_partition.txt|\\
		\verb|instance.sdf| で入力の化学グラフファイル,
	  \verb|instance_partition.txt| で分割情報の出力txtファイルを指定する.
\end{itemize}


\subsubsection{計算例}
\label{sec:instance_p}

具体例として,以下のような条件で実行する.

\begin{itemize}
\item 入力ファイル: フォルダー{\tt instances}内の{\tt sample\_1.sdf}
\item 分割情報の指定出力ファイル: {\tt partition.txt}
\end{itemize}

実行のコマンドは以下のようになる.

\bigskip

{\tt ./main ../instances/sample\_1.sdf partition.txt}

\bigskip

このコマンドを実行して場合,以下のような出力結果が得られる.

\begin{oframed}
{\bf partition.txtの内容}\\\\
{\tt 4 \\
7 \# C \\
0 0 0 \\
15 \# C \\
0 0 0 \\
10 \# C \\
0 0 0 \\
16 \# C \\
0 0 0 \\
5 \\
7 4 3 5 6 9 2 15 \# C1C1N1C1C1C1O1C \\
0 1 0 \\
7 10 \# C2C \\
0 0 0 \\
10 12 14 13 11 7 \# C1C2C1C2C1C \\
0 0 0 \\
15 16 \# C2C \\
0 0 0 \\
16 18 20 19 17 15 \# C1C2C1C2C1C \\
0 0 0 \\}
\end{oframed}

\section{構造異性体生成プログラムについて}
\label{sec: main}

\subsection{プログラムの入力と出力}
\label{sec:InOut_m}

この節では, 与えられた2-lean化学グラフの構造異性体を生成するプログラム入力と出力について説明する.
以下ではこのプログラムのことを, {\bf 異性体生成プログラム}と呼ぶ.
\ref{sec:Input_m}節では,プログラムの入力情報について説明する.
\ref{sec:Output_m}節では,プログラムの出力情報について説明する.

\subsubsection{プログラムの入力}
\label{sec:Input_m}



異性体生成プログラムでは六つの情報を入力を必要とし, 加えて一つのオプションがある.\\
一つ目は2-lean化学グラフの情報 (SDFフォーマット) である.\\
二つ目はプログラムの計算時間の上限秒である.\\
三つ目は特徴ベクトルサイズの上限である.\\
四つ目は特徴ベクトルあたりのサンプル木の数である.\\
五つ目は出力する化学グラフの個数の上限である.\\
六つ目は出力される化学グラフを保存するSDFファイル名である.\\
オプションして入力する化学グラフの分割情報である.
イプションの入力が与えた場合,最後にかくこと.
%また,入力の方法については第\ref{sec:Example}節を参照すること.

\bigskip

%\newpage
%\subsubsection{入力データの形式}
%\label{sec:InputFormat_m}

%この節では,\ref{sec:Input_m}節に説明していた入力ファイルの形式について説明する.

%\begin{oframed}
%{\bf 入力データの形式}\\\\
%\bigskip\bigskip
%\end{oframed}


\subsubsection{プログラムの出力}
\label{sec:Output_m}

異性体生成プログラムの出力は,
入力された化学グラフと同型な化学グラフの個数の下限,
生成された化学グラフの個数,
プログラムの計算時間と
入力された化学グラフと同型な化学グラフである.
化学グラフはファイル名が入力されたSDFファイルに出力される.

\bigskip

%\newpage

%\section{プログラムの説明}
%\label{sec:section4}
%この節ではプログラムの流れについて説明する.

%\newpage

\subsection{異性体生成プログラムの実行と計算例}
\label{sec:Example_m}

ここで異性体生成プログラムの実行方法と,具体例を入力した場合の計算結果を示す.

\subsubsection{実行方法}
\label{sec:compile_m}
\begin{itemize}
	\item {\em 環境確認}\\
		ISO C++ 2011標準に対応するC++コンパイラーがあれば問題ないと考えられる.
		%
		Linux Ubuntu~20.04, コンパイラー g++ ver 9.3 で確認したが,g++がインストールされていない場合は,次のようにインストールできる.\\
		\verb|$ sudo apt install g++|
	\item {\em コンパイル}\\
		\verb|$ g++ -o main main.cpp -O3 -std=c++11| 
	\item {\em 実行}\\
		\verb|$ ./main instance.txt a b c d output.sdf instance_partition.txt|\\
		\verb|instance.txt| で入力のテキストファイル,
		\verb|a| で計算時間の上限,
		\verb|b| で特徴ベクトルサイズの上限,
		\verb|c| で特徴ベクトルあたりのサンプルツリーの数,
		\verb|d| で出力する化学グラフの個数の上限,
		\verb|output.txt| で化学グラフの出力のSDFファイル,
		\verb|instance_partition.txt| で化学グラフの分割情報を指定する.
\end{itemize}


\subsubsection{計算例}
\label{sec:instance_p}

具体例として,異性体生成プログラムを以下のような条件で実行する.

\begin{itemize}
\item 入力ファイル: フォルダー{\tt instances}内の{\tt sample\_1.sdf}
\item 計算時間上限: 10秒
\item 特徴ベクトルサイズの上限: 10000000
\item 特徴ベクトルあたりのサンプルツリーの数: 5
\item 出力する化学グラフの個数上限: 2
\item 化学グラフの指定出力ファイル: {\tt output.sdf}
\item 分割情報ファイル: フォルダー{\tt instances}内の{\tt sample\_1\_partition.txt}
\end{itemize}

実行のコマンドは以下のようになる.

\bigskip


{\tt ./generate\_isomers ../instances/sample\_1.sdf 10 10000000 5 2} \\
\phantom{{\tt ./generate\_isomers ~~\,}} {\tt output.sdf ../instances/sample\_1\_partition.txt}	


\bigskip

このコマンドを実行して場合,以下のような出力結果が得られる.

\begin{oframed}
{\bf ターミナルに実行結果の具体例}\\\\
{\tt A lower bound on the number of graphs = 72\\
Number of generated graphs = 72\\
Total time : 0.00649s.}
\end{oframed}

\begin{oframed}
{\bf output.sdfの内容}\\\\
{1 \\
BH-cyclic \\
BH-cyclic \\
 20 21  0  0  0  0  0  0  0  0999 V2000 \\
    0.0000    0.0000    0.0000  C  0  0  0  0  0  0  0  0  0  0  0  0 \\
    0.0000    0.0000    0.0000  C  0  0  0  0  0  0  0  0  0  0  0  0 \\
    0.0000    0.0000    0.0000  C  0  0  0  0  0  0  0  0  0  0  0  0 \\
    0.0000    0.0000    0.0000  C  0  0  0  0  0  0  0  0  0  0  0  0 \\
    0.0000    0.0000    0.0000  N  0  0  0  0  0  0  0  0  0  0  0  0 \\
    0.0000    0.0000    0.0000  C  0  0  0  0  0  0  0  0  0  0  0  0 \\
    0.0000    0.0000    0.0000  O  0  0  0  0  0  0  0  0  0  0  0  0 \\
    0.0000    0.0000    0.0000  C  0  0  0  0  0  0  0  0  0  0  0  0 \\
    0.0000    0.0000    0.0000  O  0  0  0  0  0  0  0  0  0  0  0  0 \\
    0.0000    0.0000    0.0000  C  0  0  0  0  0  0  0  0  0  0  0  0 \\
    0.0000    0.0000    0.0000  C  0  0  0  0  0  0  0  0  0  0  0  0 \\
    0.0000    0.0000    0.0000  C  0  0  0  0  0  0  0  0  0  0  0  0 \\
    0.0000    0.0000    0.0000  C  0  0  0  0  0  0  0  0  0  0  0  0 \\
    0.0000    0.0000    0.0000  C  0  0  0  0  0  0  0  0  0  0  0  0 \\
    0.0000    0.0000    0.0000  C  0  0  0  0  0  0  0  0  0  0  0  0 \\
    0.0000    0.0000    0.0000  C  0  0  0  0  0  0  0  0  0  0  0  0 \\
    0.0000    0.0000    0.0000  C  0  0  0  0  0  0  0  0  0  0  0  0 \\
    0.0000    0.0000    0.0000  C  0  0  0  0  0  0  0  0  0  0  0  0 \\
    0.0000    0.0000    0.0000  C  0  0  0  0  0  0  0  0  0  0  0  0 \\
    0.0000    0.0000    0.0000  C  0  0  0  0  0  0  0  0  0  0  0  0 \\
  1  3  2  0  0  0  0 \\
  1  5  1  0  0  0  0 \\
  1 16  1  0  0  0  0 \\
  2  4  2  0  0  0  0 \\
  2 11  1  0  0  0  0 \\
  2 20  1  0  0  0  0 \\
  3 13  1  0  0  0  0 \\
  4 17  1  0  0  0  0 \\
  5  6  1  0  0  0  0 \\
  6  7  1  0  0  0  0 \\
  7  8  1  0  0  0  0 \\
  8  9  1  0  0  0  0 \\
  8 10  1  0  0  0  0 \\
 10 11  1  0  0  0  0 \\
 11 12  1  0  0  0  0 \\
 13 14  2  0  0  0  0 \\
 14 15  1  0  0  0  0 \\
 15 16  2  0  0  0  0 \\
 17 18  2  0  0  0  0 \\
 18 19  1  0  0  0  0 \\
 19 20  2  0  0  0  0 \\
M  END \\
\$\$\$\$ \\
2 \\
BH-cyclic \\
BH-cyclic \\
 20 21  0  0  0  0  0  0  0  0999 V2000 \\
    0.0000    0.0000    0.0000  C  0  0  0  0  0  0  0  0  0  0  0  0 \\
    0.0000    0.0000    0.0000  C  0  0  0  0  0  0  0  0  0  0  0  0 \\
    0.0000    0.0000    0.0000  C  0  0  0  0  0  0  0  0  0  0  0  0 \\
    0.0000    0.0000    0.0000  C  0  0  0  0  0  0  0  0  0  0  0  0 \\
    0.0000    0.0000    0.0000  N  0  0  0  0  0  0  0  0  0  0  0  0 \\
    0.0000    0.0000    0.0000  C  0  0  0  0  0  0  0  0  0  0  0  0 \\
    0.0000    0.0000    0.0000  O  0  0  0  0  0  0  0  0  0  0  0  0 \\
    0.0000    0.0000    0.0000  C  0  0  0  0  0  0  0  0  0  0  0  0 \\
    0.0000    0.0000    0.0000  C  0  0  0  0  0  0  0  0  0  0  0  0 \\
    0.0000    0.0000    0.0000  C  0  0  0  0  0  0  0  0  0  0  0  0 \\
    0.0000    0.0000    0.0000  C  0  0  0  0  0  0  0  0  0  0  0  0 \\
    0.0000    0.0000    0.0000  O  0  0  0  0  0  0  0  0  0  0  0  0 \\
    0.0000    0.0000    0.0000  C  0  0  0  0  0  0  0  0  0  0  0  0 \\
    0.0000    0.0000    0.0000  C  0  0  0  0  0  0  0  0  0  0  0  0 \\
    0.0000    0.0000    0.0000  C  0  0  0  0  0  0  0  0  0  0  0  0 \\
    0.0000    0.0000    0.0000  C  0  0  0  0  0  0  0  0  0  0  0  0 \\
    0.0000    0.0000    0.0000  C  0  0  0  0  0  0  0  0  0  0  0  0 \\
    0.0000    0.0000    0.0000  C  0  0  0  0  0  0  0  0  0  0  0  0 \\
    0.0000    0.0000    0.0000  C  0  0  0  0  0  0  0  0  0  0  0  0 \\
    0.0000    0.0000    0.0000  C  0  0  0  0  0  0  0  0  0  0  0  0 \\
  1  3  2  0  0  0  0 \\
  1  5  1  0  0  0  0 \\
  1 16  1  0  0  0  0 \\
  2  4  2  0  0  0  0 \\
  2 11  1  0  0  0  0 \\
  2 20  1  0  0  0  0 \\
  3 13  1  0  0  0  0 \\
  4 17  1  0  0  0  0 \\
  5  6  1  0  0  0  0 \\
  6  7  1  0  0  0  0 \\
  7  8  1  0  0  0  0 \\
  8  9  1  0  0  0  0 \\
  8 10  1  0  0  0  0 \\
 10 11  1  0  0  0  0 \\
 11 12  1  0  0  0  0 \\
 13 14  2  0  0  0  0 \\
 14 15  1  0  0  0  0 \\
 15 16  2  0  0  0  0 \\
 17 18  2  0  0  0  0 \\
 18 19  1  0  0  0  0 \\
 19 20  2  0  0  0  0 \\
M  END \\
\$\$\$\$}
\end{oframed}

\addcontentsline{toc}{section}{\refname} 
\begin{thebibliography}{10}
	\bibitem{branch}
		H.~Nagamochi and T.~Akutsu.
		A Novel Method for Inference of Chemical Compounds with Prescribed Topological Substructures Based on Integer Programming.
		Arxiv preprint, arXiv:2010.09203
\end{thebibliography}

\end{document}
