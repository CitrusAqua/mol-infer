%%%%%%%%%%%%%%%%
%
% File Module4_manual_en.txt
% English manual for the program fo rgenerating chemical isomers
% Edited: March 19, 2021, Alex
%
%%%%%%%%%%%%%%%%%%%%%%%

\documentclass[11pt,titlepage,dvipdfmx,twoside]{article}
\linespread{1.1}

\usepackage{amsfonts}
\usepackage{amssymb}
\usepackage{amsmath}
\usepackage{amsthm}
\newtheorem{theorem}{Theorem}
\newtheorem{lemma}[theorem]{Lemma}
\usepackage{enumitem}
\usepackage{geometry}
\geometry{left=2.5cm,right=2.5cm,top=2.5cm,bottom=2.5cm}

\usepackage{algorithm}  
\usepackage{algorithmic}  
\renewcommand{\algorithmicrequire}{\textbf{Input:}} 
\renewcommand{\algorithmicensure}{\textbf{Output:}}
\renewcommand{\algorithmicforall}{\textbf{for each}}

\usepackage{mathtools}
\usepackage{comment}
\usepackage[dvipdfmx]{graphicx}
\usepackage{float}
\usepackage{framed}
\usepackage{graphicx}
\usepackage{subcaption}
\usepackage{color}
\usepackage{url}

\usepackage{dirtree}

\newcommand{\project}{{\tt mol-infer/Cyclic}}


\title{\huge{
Module 4:
Listing Chemical Isomers of a Given 2-Lean Cyclic Chemical Graph}}

\author{\project}

\begin{document}

% % The following makeatletter must be after begin{document}
% \makeatletter 
% \let\c@lstlisting\c@figure
% \makeatother

% \西暦
\date{\today}

\maketitle

% \cleardoublepage

\thispagestyle{empty}
\tableofcontents
\clearpage

\pagenumbering{arabic}


\section{Introduction}
\label{sec:intro}

This text explains how to use the program for listing
chemical isomers of a given 2-lean cyclic chemical graph~\cite{branch}.

The structure of files and folders of this module are 
as follows\\
\dirtree{%
.1 ./Module\_4.
.2 Manual\_Module\_4\_Cyclic\_en.pdf.
.2 Manual\_Module\_4\_Cyclic\_en.tex.
.2 Manual\_Module\_4\_Cyclic\_jp.pdf.
.2 Manual\_Module\_4\_Cyclic\_jp.tex.
.2 Pseudocode\_Graph\_Generation.pdf.
.2 files.
.3 Makefile.
.3 executables.
.4 linux.
.5 generate\_isomers.
.5 generate\_partition.
.4 osx.
.5 generate\_isomers.
.5 generate\_partition.
.4 windows.
.5 generate\_partition.exe.
.5 generate\_isomers.exe.
.3 instances.
.4 sample\_1\_partition.txt.
.4 sample\_1.png.
.4 sample\_1.sdf.
.3 main.
.4 readme.txt.
.4 generate\_partition.cpp.
.4 output.sdf.
.4 generate\_isomers.cpp.
.3 include.
.4 chemical\_graph.hpp.
.4 data\_structures.hpp.
.4 tools.hpp.
.4 cross\_timer.h.
.4 fringe\_tree.hpp.
}

\smallskip
The contents of each of the files is as follows.
%
\begin{itemize}		
 \item {\tt Manual\_Module\_4\_Cyclic\_en.pdf}\\
 This file.
 \item {\tt Manual\_Module\_4\_Cyclic\_en.tex}\\
 A \LaTeX~source file of this manual.
 \item {\tt Manual\_Module\_4\_Cyclic\_jp.pdf}\\
 The Japanese version of this manual.
 \item {\tt Manual\_Module\_4\_Cyclic\_jp.tex}\\
 A \LaTeX~source file of the Japanese version of the manual.
\item{ {\tt Pseudocode\_Graph\_Generation.pdf}}\\
  A pdf file with Pseudo-codes for the graph search algorithm.
 \item Folder {\tt files}
 \begin{itemize}
 \item {\tt Makefile} \\
	    Compilation directives for the C++ programs of this module.
    \item Folder {\tt executables}\\
    A folder that contains compiled executable files for 
    each of the architectures: linux, osx, and windows.
    
	\item Folder {\tt instances}\\
		A folder containing sample input instances.
		\begin{itemize}
			\item{\tt sample\_1.sdf}\\
			  A cyclic chemical graph with 20 vertices (non-Hydrogen atoms),
			    core size 18 and core height~1.
			\item{\tt sample\_1\_partition.txt}\\
			  A file containing partition information into acyclic subgraphs 
				of the cyclic chemical graph given in {\tt sample\_1.sdf}.
			\item{\tt sample\_2.sdf}\\
			  A cyclic chemical graph with 50 vertices, 
			    core size 24 and core height~6.
			\item{\tt sample\_2\_partition.txt}\\
			  A file containing partition information into acyclic subgraphs 
				of the cyclic chemical graph given in {\tt sample\_2.sdf}.
			\item{\tt sample\_3.sdf}\\
			  A cyclic chemical graph with 60 vertices, 
			    core size 31 and core height~4.
			\item{\tt sample\_3\_partition.txt}\\
			  A file containing partition information into acyclic subgraphs 
				of the cyclic chemical graph given in {\tt sample\_3.sdf}.
			\item{\tt sample\_4.sdf}\\
			  A cyclic chemical graph with 120 vertices, 
			    core size 60 and core height~4.
			\item{\tt sample\_4\_partition.txt}\\
			  A file containing partition information into acyclic subgraphs 
				of the cyclic chemical graph given in {\tt sample\_4.sdf}.
		\end{itemize}
    \item Folder {\tt main}\\
	    Folder containing source files written in the C++ programming language.
	    \begin{itemize}
		    \item{ \tt generate\_partition.cpp}\\
			    Functions for calculating a partition of a cyclic chemical graph into
			    acyclic subgraphs.
			    
		    \item{ \tt generate\_isomers.cpp}\\
			    Implements an algorithm for listing chemical isomers of
			    a 2-lean cyclic chemical graph.

	\item Folder {\tt include}\\
		A folder that contains related header files, written in C++.
		\begin{itemize}
			\item{chemical\_graph.hpp}\\
			  A header file that contains functions for manipulating 
			  chemical graphs.
			\item{cross\_timer.h}\\
				A header file that contains functions for measuring execution time.
				
			\item{data\_structures.hpp}\\
				Data structures implemented for storing chemical graphs.
				
			\item{debug.h}\\
				Used for debugging purposes.
				
			\item{fringe\_tree.hpp}\\
				Header file with functions for enumerating 2-fringe trees~\cite{branch}.
				
			\item{tools.hpp}\\
				Various functions used in the implementation.
		\end{itemize}
		\end{itemize}
	  \end{itemize}
      \end{itemize}

The remainder of this text is organized as follows.
Section~\ref{sec:term} explains some of the terminology used throughout the text.
Section~\ref{sec: partition} gives an explanation of the input and the output of
the program for calculating a partition of a cyclic chemical graph into
acyclic subgraphs, and provides a computational example.
Section~\ref{sec: main}
gives an explanation of the program for generating chemical isomers of
a given 2-lean cyclic chemical graph, explaining the input, output, and
presents a computational example.


% \newpage

\section{Terminology}
\label{sec:term}

This section gives an overview of the terminology used in the text.
%
\begin{itemize}
%
\item Chemical Graph\\
A graph-theoretical description of a chemical compound, 
where the graph's vertices correspond to atoms, and
its (multi) edges to chemical bonds.
Each vertex is colored with the chemical element of the atom it corresponds to,
and edges have multiplicity according to the corresponding bond order.
We deal with ``hydrogen-suppressed'' graphs, 
where none of the graph's vertices is colored as hydrogen.
This can be done without loss of generality,
since there is a unique way to saturate a hydrogen-suppressed chemical graph 
with hydrogen atoms subject to a fixed valence of each chemical element.


\item Feature vector\\
A numerical vector giving information such as the count of
each chemical element in a chemical graph.
For a complete information on the descriptors used in feature vectors for
this project, please see~\cite{branch}.



\item Partition Information\\
Information necessary to specify the base vertices and edges, as well as the vertex and edge components
of a chemical graph.
For more details, please check~\cite{branch}.


% \item branch-height, core size, core height, 
% ヘリ木(fringe tree),
% 核(core) ,
% 基底頂点・辺(base vertex/edge)
% 内部・外部節点(internal/external vertex),
% 内部・外部原子結合(internal/external adjacency-configuration),
% について参考文献\cite{branch}に参考すること.

\end{itemize}

\section{Program for Calculating a Partition Into Acyclic Subgraphs}
\label{sec: partition}

\subsection{Input and Output}
\label{sec:InOut_p}

This section explains the input and output information of the 
program that is used to calculate a partition of a cyclic chemical graph
into acyclic subgraphs.
We call this program {\tt Partition}.
Following,
Sec.~\ref{sec:Input_p} explains the input format, and
Sec.~\ref{sec:Output_p} gives an explanation the output information of the program.


\subsubsection{The Program's Input}
\label{sec:Input_p}


The program {\tt Partition} takes two items as its input.

First, is a cyclic chemical graph, given as a ``structured data file,'' in  SDF format.
This is a standard format for representing chemical graphs.
For more details, please check the documentation at \\
\url{ http://help.accelrysonline.com/ulm/onelab/1.0/content/ulm_pdfs/direct/reference/} \\
\phantom{\url{ http://help.accelrysonline.com}}\url{/ctfileformats2016.pdf} 

Next,  is a filename of a text file where the information on 
a partition of the graph given in the SDF into acyclic subgraphs
calculated by the program will be saved.


\subsubsection{The Program's Output}
\label{sec:Output_p}

The output of this program is a 
partition into acyclic subgraphs of the cyclic chemical graph given in
the input.
The partition information is stored as a text file with filename as provided
in the input parameters.


\begin{oframed}
{\bf Output Format Example}\\\\
{\tt 4 \\
7 \# C \\
0 0 0 \\
15 \# C \\
0 0 0 \\
10 \# C \\
0 0 0 \\
16 \# C \\
0 0 0 \\
5 \\
7 4 3 5 6 9 2 15 \# C1C1N1C1C1C1O1C \\
0 1 0 \\
7 10 \# C2C \\
0 0 1 \\
10 12 14 13 11 7 \# C1C2C1C2C1C \\
0 0 1 \\
15 16 \# C2C \\
0 0 1 \\
16 18 20 19 17 15 \# C1C2C1C2C1C \\
0 0 1 \\}
\end{oframed}


Following, Table~\ref{tab:PartitionFormat}
gives a row-by-row explanation of the numerical information in the
above output file.


\bigskip
\begin{table}[H]
\begin{center} \caption{Structure of an Output File with Partition Information}
\label{tab:PartitionFormat}
  \begin{tabular}{l|l}
  Value in the file & Explanation \\ \hline \hline
{\tt  4} & Number of base vertices \\ \hline
{\tt  7 \# C} & The index of a base vertex in the input SDF and its element\\
{\tt  0 0 0} & Core height lower and upper bound; \\
{\tt  15 \# C} & \hspace{10mm} the vertex component can be created or not (0/1) \\
{\tt  0 0 0} & \\
{\tt  10 \# C} & \\
{\tt  0 0 0} & \\ 
{\tt  16 \# C} &\\ 
{\tt  0 0 0} & \\ \hline
{\tt  5} & Number of base edges \\ \hline
{\tt  7 4 3 5 6 9 2 15 \# C1C1N1C1C1C1O1C} & Indices of vertices in the base edge from the input SDF,\\
{\tt  0 1 0} & \hspace{10mm} their elements and bond multiplicities \\
{\tt  7 10 \# C2C} &\\ 
{\tt  0 0 1} &Core height lower and upper bound,\\
{\tt  10 12 14 13 11 7 \# C1C2C1C2C1C} &\hspace{10mm} the edge component can be created or not (0/1) \\
{\tt  0 0 1} & \\
{\tt  15 16 \# C2C} & \\
{\tt  0 0 1} & \\
{\tt  16 18 20 19 17 15 \# C1C2C1C2C1C} & \\
{\tt  0 0 1} & \\ \hline
  \end{tabular}
\end{center}
\end{table}



\subsection{Program Execution and Computation Example}
\label{sec:Example_p}

This section gives an explanation on how to compile and run the program,
as well as a concrete computational example of the program's execution.


\subsubsection{
Compiling and Executing the Program for Generating
a Decomposition of a Cyclic Graph  into Acyclic Graphs
}
\label{sec:compile_p}

A compiled executable that has been tested on\\
- linux \\
- osx \\
- windows (cygwin) \\
is included in the set of files.
In addition, we describe how the accompanying source files can be compiled.

\begin{itemize}
	\item {\em Computation environment}\\
		There should not be any problems when using an ISO C++11 compatible compiler. %
% 		The program has been tested on 
% 		Ubuntu Linux 20.04,  with the  g++ compiler version 9.3.
% 		If the compiler is not installed on the system, it can be installed with
% 		the following command.\\
% 		\verb|$ sudo apt install g++|
	\item {\em Compiling the program}\\
	      In the terminal, navigate to the {\tt files} subfolder.
	      Then, if the {\tt make} command is available on the system, the program can be simply compiled by typing \\
	      \verb|$ make generate_partition|\\
	      In case the {\tt make} command is not available, then the program can be compiled as \\	      
		\verb|$ g++ -o generate_partition ./main/generate_partition.cpp -O3 -std=c++11|\\
	\item {\em Executing the program}\\
		\verb|$ ./generate_partition instance.sdf instance_partition.txt|\\
		\verb|instance.sdf|  is an input SDF, 
		and a partition information as the output of the program is stored in the file
	  \verb|instance_partition.txt| .
\end{itemize}


\subsubsection{Computational Example}
\label{sec:instance_p}

This section illustrates a concrete computational 
example of running the {\tt Partition} program.
We assume the following:
%
\begin{itemize}
\item Input file: {\tt sample\_1.sdf} from the folder {\tt instances} (see Sec.~\ref{sec:intro})
\item Output file to store the partition information: {\tt partition.txt}
\end{itemize}

Run the following command in the terminal to execute the program.

\bigskip

{\tt ./generate\_partition ./instances/sample\_1.sdf partition.txt}

\bigskip

After successfully executing the program, the contents of the file 
{\tt partition.txt} should be as follows.

\begin{oframed}
{\bf Contents of the file {\tt partition.txt}}\\\\
{\tt 4 \\
7 \# C \\
0 0 0 \\
15 \# C \\
0 0 0 \\
10 \# C \\
0 0 0 \\
16 \# C \\
0 0 0 \\
5 \\
7 4 3 5 6 9 2 15 \# C1C1N1C1C1C1O1C \\
0 1 0 \\
7 10 \# C2C \\
0 0 0 \\
10 12 14 13 11 7 \# C1C2C1C2C1C \\
0 0 0 \\
15 16 \# C2C \\
0 0 0 \\
16 18 20 19 17 15 \# C1C2C1C2C1C \\
0 0 0 \\}
\end{oframed}


\section{Program for Generating Chemical Isomers}
\label{sec: main}

\subsection{Input and Output of the Program}
\label{sec:InOut_m}

This section gives an explanation of the input and output
of the program that generates chemical isomers
of a given 2-lean chemical graph.
We call the program {\tt generate\_isomers}.
Section~\ref{sec:Input_m} gives an explanation of the program's input, and
Sec.~\ref{sec:Output_m} of the program's output.

\subsubsection{The Program's Input}
\label{sec:Input_m}


The input to the {\tt generate\_isomers} program
consists of six necessary and one optional item. \\
First, comes information of a 2-lean chemical graph (in SDF format).\\
Second, comes a time limit in seconds on each of the stages of the program 
(for details, check the accompanying file with pseudo-codes of the algorithm).\\
Third is an upper bound on the number of {\em partial} feature vectors that the program stores
during its computation. \\
Fourth is the number of ``sample graphs'' that are stored per one feature vector.\\
Fifth is an upper limit on the number of generated output chemical graphs. \\
Sixth is a filename (SDF) where the output graphs will be written.\\
An optional parameter is a filename with a partition information of
the chemical graph given as the first parameter, and if given, it must be last
in the list of parameters.



\subsubsection{The Program's Output}
\label{sec:Output_m}

After executing the {\tt generate\_isomers} program,
the chemical isomers of the input graph will be written
in the specified SDF, and some information on the execution will be output on 
the terminal.
The information printed on the terminal includes:\\
 - a lower bound on the number of chemical isomers of the given input chemical graph, \\
 - the number of graphs that the program generated under the given parameters, and \\
 - the program's execution time.
 


\subsection{Executing the Program and a Computational Example}
\label{sec:Example_m}

This section gives a concrete computational example of the {\tt generate\_isomers} program.


\subsubsection{Compiling and Executing the {\tt generate\_isomers} Program}
\label{sec:compile_m}

A compiled executable that has been tested on \\
- linux \\
- osx \\
- windows (cygwin) \\
is included in the set of files.
In addition, we describe how the accompanying source files can be compiled.

\begin{itemize}
	\item {\em Computation environment}\\
		There should not be any problems when using a ISO C++ compatible compiler. %
% 		The program has been tested on 
% 		Ubuntu Linux 20.04,  with the  g++ compiler version 9.3.
% 		If the compiler is not installed on the system, it can be installed with
% 		the following command.\\
% 		\verb|$ sudo apt install g++|
	\item {\em Compiling the program}\\
		If the {\tt make} command is available on the system, 
		then the program can be compiled in the {\tt files} folder by typing \\
		\verb|$ make generate_isomers|\\
		in the terminal.
		If the {\tt make} command is not available, then
		please run the following command in the terminal.\\
		\verb|$ g++ -o generate_isomers ./main/generate_isomers.cpp -O3 -std=c++11|\\
	\item {\em Executing the program}\\
		The program can be executed by running the following command in the terminal.\\
		\verb|$ ./generate_isomers instance.txt a b c d output.sdf instance_partition.txt|\\
		Above, {\tt generate\_isomers} is the name of the program's executable file, and the remaining command-line
		parameters are as follows: \\
		\verb|instance.txt|  a text file containing a chemical specification \\
		\verb|a| upper bound (in seconds) on the computation time, \\
		\verb|b| upper bound on the number of stored partial feature vectors, \\
		\verb|c| upper bound on the number of sample graphs stored per feature vector, \\
		\verb|d| upper bound on the number of output graphs, \\
		\verb|output.sdf| filename to store the output chemical graphs (SDF format), \\
		\verb|instance_partition.txt|  partition information of the input chemical graph.
\end{itemize}


\subsubsection{Computational Example}
\label{sec:instance_p}

We execute the {\tt generate\_isomers} program with the following parameters.

\begin{itemize}
\item Input graph: File {\tt sample\_1.sdf} from the folder {\tt instances} (see Sec.~\ref{sec:intro})
\item Time limit: 10 seconds
\item Upper limit on the number of partial feature vectors: 10000000
\item Number of sample graphs per feature vector: 5
\item Upper limit on the number of output graphs: 2
\item Filename to store the output graphs: {\tt output.sdf}
\item Partition information of the input graph: File {\tt sample\_1\_partition.txt} from the folder {\tt instances}.
\end{itemize}

Execute the program by typing the following command into the terminal (without a line break).

\bigskip


{\tt ./generate\_isomers ./instances/sample\_1.sdf 10 10000000 5 2} \\
\phantom{{\tt ./generate\_isomers ~~\,}} {\tt output.sdf ./instances/sample\_1\_partition.txt}	


\bigskip

Upon successful execution of the program, the following text should appear on the terminal.

\begin{oframed}
{\bf Output Written on the Terminal}\\\\
{\tt A lower bound on the number of graphs = 72\\
Number of generated graphs = 2\\
Total time : 0.00649s.}
\end{oframed}

\begin{oframed}
{\bf Contents of the file output.sdf}\\\\
{1 \\
BH-cyclic \\
BH-cyclic \\
 20 21  0  0  0  0  0  0  0  0999 V2000 \\
    0.0000    0.0000    0.0000  C  0  0  0  0  0  0  0  0  0  0  0  0 \\
    0.0000    0.0000    0.0000  C  0  0  0  0  0  0  0  0  0  0  0  0 \\
    0.0000    0.0000    0.0000  C  0  0  0  0  0  0  0  0  0  0  0  0 \\
    0.0000    0.0000    0.0000  C  0  0  0  0  0  0  0  0  0  0  0  0 \\
    0.0000    0.0000    0.0000  N  0  0  0  0  0  0  0  0  0  0  0  0 \\
    0.0000    0.0000    0.0000  C  0  0  0  0  0  0  0  0  0  0  0  0 \\
    0.0000    0.0000    0.0000  O  0  0  0  0  0  0  0  0  0  0  0  0 \\
    0.0000    0.0000    0.0000  C  0  0  0  0  0  0  0  0  0  0  0  0 \\
    0.0000    0.0000    0.0000  O  0  0  0  0  0  0  0  0  0  0  0  0 \\
    0.0000    0.0000    0.0000  C  0  0  0  0  0  0  0  0  0  0  0  0 \\
    0.0000    0.0000    0.0000  C  0  0  0  0  0  0  0  0  0  0  0  0 \\
    0.0000    0.0000    0.0000  C  0  0  0  0  0  0  0  0  0  0  0  0 \\
    0.0000    0.0000    0.0000  C  0  0  0  0  0  0  0  0  0  0  0  0 \\
    0.0000    0.0000    0.0000  C  0  0  0  0  0  0  0  0  0  0  0  0 \\
    0.0000    0.0000    0.0000  C  0  0  0  0  0  0  0  0  0  0  0  0 \\
    0.0000    0.0000    0.0000  C  0  0  0  0  0  0  0  0  0  0  0  0 \\
    0.0000    0.0000    0.0000  C  0  0  0  0  0  0  0  0  0  0  0  0 \\
    0.0000    0.0000    0.0000  C  0  0  0  0  0  0  0  0  0  0  0  0 \\
    0.0000    0.0000    0.0000  C  0  0  0  0  0  0  0  0  0  0  0  0 \\
    0.0000    0.0000    0.0000  C  0  0  0  0  0  0  0  0  0  0  0  0 \\
  1  3  2  0  0  0  0 \\
  1  5  1  0  0  0  0 \\
  1 16  1  0  0  0  0 \\
  2  4  2  0  0  0  0 \\
  2 11  1  0  0  0  0 \\
  2 20  1  0  0  0  0 \\
  3 13  1  0  0  0  0 \\
  4 17  1  0  0  0  0 \\
  5  6  1  0  0  0  0 \\
  6  7  1  0  0  0  0 \\
  7  8  1  0  0  0  0 \\
  8  9  1  0  0  0  0 \\
  8 10  1  0  0  0  0 \\
 10 11  1  0  0  0  0 \\
 11 12  1  0  0  0  0 \\
 13 14  2  0  0  0  0 \\
 14 15  1  0  0  0  0 \\
 15 16  2  0  0  0  0 \\
 17 18  2  0  0  0  0 \\
 18 19  1  0  0  0  0 \\
 19 20  2  0  0  0  0 \\
M  END \\
\$\$\$\$ \\
2 \\
BH-cyclic \\
BH-cyclic \\
 20 21  0  0  0  0  0  0  0  0999 V2000 \\
    0.0000    0.0000    0.0000  C  0  0  0  0  0  0  0  0  0  0  0  0 \\
    0.0000    0.0000    0.0000  C  0  0  0  0  0  0  0  0  0  0  0  0 \\
    0.0000    0.0000    0.0000  C  0  0  0  0  0  0  0  0  0  0  0  0 \\
    0.0000    0.0000    0.0000  C  0  0  0  0  0  0  0  0  0  0  0  0 \\
    0.0000    0.0000    0.0000  N  0  0  0  0  0  0  0  0  0  0  0  0 \\
    0.0000    0.0000    0.0000  C  0  0  0  0  0  0  0  0  0  0  0  0 \\
    0.0000    0.0000    0.0000  O  0  0  0  0  0  0  0  0  0  0  0  0 \\
    0.0000    0.0000    0.0000  C  0  0  0  0  0  0  0  0  0  0  0  0 \\
    0.0000    0.0000    0.0000  C  0  0  0  0  0  0  0  0  0  0  0  0 \\
    0.0000    0.0000    0.0000  C  0  0  0  0  0  0  0  0  0  0  0  0 \\
    0.0000    0.0000    0.0000  C  0  0  0  0  0  0  0  0  0  0  0  0 \\
    0.0000    0.0000    0.0000  O  0  0  0  0  0  0  0  0  0  0  0  0 \\
    0.0000    0.0000    0.0000  C  0  0  0  0  0  0  0  0  0  0  0  0 \\
    0.0000    0.0000    0.0000  C  0  0  0  0  0  0  0  0  0  0  0  0 \\
    0.0000    0.0000    0.0000  C  0  0  0  0  0  0  0  0  0  0  0  0 \\
    0.0000    0.0000    0.0000  C  0  0  0  0  0  0  0  0  0  0  0  0 \\
    0.0000    0.0000    0.0000  C  0  0  0  0  0  0  0  0  0  0  0  0 \\
    0.0000    0.0000    0.0000  C  0  0  0  0  0  0  0  0  0  0  0  0 \\
    0.0000    0.0000    0.0000  C  0  0  0  0  0  0  0  0  0  0  0  0 \\
    0.0000    0.0000    0.0000  C  0  0  0  0  0  0  0  0  0  0  0  0 \\
  1  3  2  0  0  0  0 \\
  1  5  1  0  0  0  0 \\
  1 16  1  0  0  0  0 \\
  2  4  2  0  0  0  0 \\
  2 11  1  0  0  0  0 \\
  2 20  1  0  0  0  0 \\
  3 13  1  0  0  0  0 \\
  4 17  1  0  0  0  0 \\
  5  6  1  0  0  0  0 \\
  6  7  1  0  0  0  0 \\
  7  8  1  0  0  0  0 \\
  8  9  1  0  0  0  0 \\
  8 10  1  0  0  0  0 \\
 10 11  1  0  0  0  0 \\
 11 12  1  0  0  0  0 \\
 13 14  2  0  0  0  0 \\
 14 15  1  0  0  0  0 \\
 15 16  2  0  0  0  0 \\
 17 18  2  0  0  0  0 \\
 18 19  1  0  0  0  0 \\
 19 20  2  0  0  0  0 \\
M  END \\
\$\$\$\$
}
\end{oframed}

\addcontentsline{toc}{section}{\refname} 
\begin{thebibliography}{10}
	\bibitem{branch}
	  T.~Akutsu and H.~Nagamochi.
	  A Novel Method for Inference of Chemical Compounds with Prescribed Topological Substructures Based on Integer Programming.
	  Arxiv preprint, arXiv:2010.09203
\end{thebibliography}

\end{document}
